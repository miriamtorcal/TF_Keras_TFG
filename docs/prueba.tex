\documentclass[a4paper,12pt,twoside]{memoir}

% Castellano
\usepackage[spanish,es-tabla]{babel}
\selectlanguage{spanish}
\usepackage[utf8]{inputenc}
\usepackage[T1]{fontenc}
\usepackage{lmodern} % Scalable font
\usepackage{microtype}
\usepackage{placeins}
\usepackage{amssymb} % Símbolo de verificación

\RequirePackage{booktabs}
\RequirePackage[table]{xcolor}
\RequirePackage{xtab}
\RequirePackage{multirow}

% Links
\PassOptionsToPackage{hyphens}{url}\usepackage[colorlinks]{hyperref}
\hypersetup{
	allcolors = {red}
}

% Ecuaciones
\usepackage{amsmath}

% Rutas de fichero / paquete
\newcommand{\ruta}[1]{{\sffamily #1}}

% Párrafos
\nonzeroparskip

% Huérfanas y viudas
\widowpenalty100000
\clubpenalty100000

% Imagenes
\usepackage{graphicx}
\newcommand{\imagen}[2]{
	\begin{figure}[!h]
		\centering
		\includegraphics[width=0.9\textwidth]{#1}
		\caption{#2}\label{fig:#1}
	\end{figure}
	\FloatBarrier
}

\newcommand{\imagenflotante}[2]{
	\begin{figure}%[!h]
		\centering
		\includegraphics[width=0.9\textwidth]{#1}
		\caption{#2}\label{fig:#1}
	\end{figure}
}

\newcommand{\imagengrande}[3]{
	\begin{figure}[!h]
		\centering
		\includegraphics[width=#1\textwidth]{#2}
		\caption{#3}\label{fig:#2}
	\end{figure}
}

% El comando \figura nos permite insertar figuras comodamente, y utilizando
% siempre el mismo formato. Los parametros son:
% 1 -> Porcentaje del ancho de página que ocupará la figura (de 0 a 1)
% 2 --> Fichero de la imagen
% 3 --> Texto a pie de imagen
% 4 --> Etiqueta (label) para referencias
% 5 --> Opciones que queramos pasarle al \includegraphics
% 6 --> Opciones de posicionamiento a pasarle a \begin{figure}
\newcommand{\figuraConPosicion}[6]{%
  \setlength{\anchoFloat}{#1\textwidth}%
  \addtolength{\anchoFloat}{-4\fboxsep}%
  \setlength{\anchoFigura}{\anchoFloat}%
  \begin{figure}[#6]
    \begin{center}%
      \Ovalbox{%
        \begin{minipage}{\anchoFloat}%
          \begin{center}%
            \includegraphics[width=\anchoFigura,#5]{#2}%
            \caption{#3}%
            \label{#4}%
          \end{center}%
        \end{minipage}
      }%
    \end{center}%
  \end{figure}%
}

%
% Comando para incluir imágenes en formato apaisado (sin marco).
\newcommand{\figuraApaisadaSinMarco}[5]{%
  \begin{figure}%
    \begin{center}%
    \includegraphics[angle=90,height=#1\textheight,#5]{#2}%
    \caption{#3}%
    \label{#4}%
    \end{center}%
  \end{figure}%
}
% Para las tablas
\newcommand{\otoprule}{\midrule [\heavyrulewidth]}
%
% Nuevo comando para tablas pequeñas (menos de una página).
\newcommand{\tablaSmall}[5]{%
 \begin{table}[!h]
  \begin{center}
   \rowcolors {2}{gray!35}{}
   \begin{tabular}{#2}
    \toprule
    #4
    \otoprule
    #5
    \bottomrule
   \end{tabular}
   \caption{#1}
   \label{tabla:#3}
  \end{center}
 \end{table}
}

%
% Nuevo comando para tablas pequeñas (menos de una página).
\newcommand{\tablaSmallSinColores}[5]{%
 \begin{table}[H]
  \begin{center}
   \begin{tabular}{#2}
    \toprule
    #4
    \otoprule
    #5
    \bottomrule
   \end{tabular}
   \caption{#1}
   \label{tabla:#3}
  \end{center}
 \end{table}
}

\newcommand{\tablaApaisadaSmall}[5]{%
\begin{landscape}
  \begin{table}
   \begin{center}
    \rowcolors {2}{gray!35}{}
    \begin{tabular}{#2}
     \toprule
     #4
     \otoprule
     #5
     \bottomrule
    \end{tabular}
    \caption{#1}
    \label{tabla:#3}
   \end{center}
  \end{table}
\end{landscape}
}

%
% Nuevo comando para tablas grandes con cabecera y filas alternas coloreadas en gris.
\newcommand{\tabla}[6]{%
  \begin{center}
    \tablefirsthead{
      \toprule
      #5
      \otoprule
    }
    \tablehead{
      \multicolumn{#3}{l}{\small\sl continúa desde la página anterior}\\
      \toprule
      #5
      \otoprule
    }
    \tabletail{
      \hline
      \multicolumn{#3}{r}{\small\sl continúa en la página siguiente}\\
    }
    \tablelasttail{
      \hline
    }
    \bottomcaption{#1}
    \rowcolors {2}{gray!35}{}
    \begin{xtabular}{#2}
      #6
      \bottomrule
    \end{xtabular}
    \label{tabla:#4}
  \end{center}
}

%
% Nuevo comando para tablas grandes con cabecera.
\newcommand{\tablaSinColores}[6]{%
  \begin{center}
    \tablefirsthead{
      \toprule
      #5
      \otoprule
    }
    \tablehead{
      \multicolumn{#3}{l}{\small\sl continúa desde la página anterior}\\
      \toprule
      #5
      \otoprule
    }
    \tabletail{
      \hline
      \multicolumn{#3}{r}{\small\sl continúa en la página siguiente}\\
    }
    \tablelasttail{
      \hline
    }
    \bottomcaption{#1}
    \begin{xtabular}{#2}
      #6
      \bottomrule
    \end{xtabular}
    \label{tabla:#4}
  \end{center}
}

%
% Nuevo comando para tablas grandes sin cabecera.
\newcommand{\tablaSinCabecera}[5]{%
  \begin{center}
    \tablefirsthead{
      \toprule
    }
    \tablehead{
      \multicolumn{#3}{l}{\small\sl continúa desde la página anterior}\\
      \hline
    }
    \tabletail{
      \hline
      \multicolumn{#3}{r}{\small\sl continúa en la página siguiente}\\
    }
    \tablelasttail{
      \hline
    }
    \bottomcaption{#1}
  \begin{xtabular}{#2}
    #5
   \bottomrule
  \end{xtabular}
  \label{tabla:#4}
  \end{center}
}



\definecolor{cgoLight}{HTML}{EEEEEE}
\definecolor{cgoExtralight}{HTML}{FFFFFF}

%
% Nuevo comando para tablas grandes sin cabecera.
\newcommand{\tablaSinCabeceraConBandas}[5]{%
  \begin{center}
    \tablefirsthead{
      \toprule
    }
    \tablehead{
      \multicolumn{#3}{l}{\small\sl continúa desde la página anterior}\\
      \hline
    }
    \tabletail{
      \hline
      \multicolumn{#3}{r}{\small\sl continúa en la página siguiente}\\
    }
    \tablelasttail{
      \hline
    }
    \bottomcaption{#1}
    \rowcolors[]{1}{cgoExtralight}{cgoLight}

  \begin{xtabular}{#2}
    #5
   \bottomrule
  \end{xtabular}
  \label{tabla:#4}
  \end{center}
}



\graphicspath{ {./img/} }

% Capítulos
\chapterstyle{bianchi}
\newcommand{\capitulo}[2]{
	\setcounter{chapter}{#1}
	\setcounter{section}{0}
	\setcounter{figure}{0}
	\setcounter{table}{0}
	\chapter*{#2}
	\addcontentsline{toc}{chapter}{#2}
	\markboth{#2}{#2}
}

% Apéndices
\renewcommand{\appendixname}{Apéndice}
\renewcommand*\cftappendixname{\appendixname}

\newcommand{\apendice}[1]{
	%\renewcommand{\thechapter}{A}
	\chapter{#1}
}

\renewcommand*\cftappendixname{\appendixname\ }

% Formato de portada
\makeatletter
\usepackage{xcolor}
\newcommand{\tutor}[1]{\def\@tutor{#1}}
\newcommand{\course}[1]{\def\@course{#1}}
\definecolor{cpardoBox}{HTML}{E6E6FF}
\def\maketitle{
  \null
  \thispagestyle{empty}
  % Cabecera ----------------
\noindent\includegraphics[width=\textwidth]{cabecera}\vspace{1cm}%
  \vfill
  % Título proyecto y escudo informática ----------------
  \colorbox{cpardoBox}{%
    \begin{minipage}{.8\textwidth}
      \vspace{.5cm}\Large
      \begin{center}
      \textbf{TFG del Grado en Ingeniería Informática}\vspace{.6cm}\\
      \textbf{\LARGE\@title{}}
      \end{center}
      \vspace{.2cm}
    \end{minipage}

  }%
  \hfill\begin{minipage}{.20\textwidth}
    \includegraphics[width=\textwidth]{escudoInfor}
  \end{minipage}
  \vfill
  % Datos de alumno, curso y tutores ------------------
  \begin{center}%
  {%
    \noindent\LARGE
    Presentado por \@author{}\\ 
    en Universidad de Burgos --- \@date{}\\
    Tutor: \@tutor{}\\
  }%
  \end{center}%
  \null
  \cleardoublepage
  }
\makeatother

\newcommand{\nombre}{Francisco Martín Vargas} %%% cambio de comando
\newcommand{\nombretutor}{César Represa Pérez}

% Datos de portada
\title{{\Huge DomoCamera}\\[0.25cm]Entorno domótico para la monitorización de cámaras desde Android.}
\author{\nombre}
\tutor{\nombretutor}
\date{\today}

\begin{document}

\maketitle


\newpage\null\thispagestyle{empty}\newpage


%%%%%%%%%%%%%%%%%%%%%%%%%%%%%%%%%%%%%%%%%%%%%%%%%%%%%%%%%%%%%%%%%%%%%%%%%%%%%%%%%%%%%%%%
\thispagestyle{empty}


\noindent\includegraphics[width=\textwidth]{cabecera}\vspace{1cm}

\noindent D. \nombretutor, profesor del departamento de Ingeniería Electromecánica, área de Tecnología Electrónica.

\noindent Expone:

\noindent Que el alumno D. \nombre, con DNI 71704736M, ha realizado el Trabajo final de Grado en Ingeniería Informática titulado ``DomoCamera --- Entorno domótico para la monitorización de cámaras desde Android''.

\noindent Y que dicho trabajo ha sido realizado por el alumno bajo la dirección del que suscribe, en virtud de lo cual se autoriza su presentación y defensa.

\begin{center} %\large
En Burgos, {\large \today}
\end{center}

\vfill\vfill\vfill

% Author and supervisor
Vº. Bº. del Tutor:\\[2cm]
D. \nombretutor


\newpage\null\thispagestyle{empty}\newpage




\frontmatter

% Abstract en castellano
\renewcommand*\abstractname{Resumen}
\begin{abstract}
El auge de la domótica y los teléfonos móviles ha tenido un impacto en la sociedad. Por un lado, la domótica facilita una gestión más eficiente en el hogar y hace más segura una vivienda. Por otro lado, la telefonía móvil pone a nuestro alcance cientos de herramientas tecnológicas  que mejoran la calidad de vida de las personas. 

Este proyecto muestra la unión perfecta entre estos dos sectores, poniéndolos además al alcance de todo el mundo para poder gestionar la seguridad del hogar desde nuestro teléfono móvil. Para ello se pretende crear un entorno que nos permita monitorizar cámaras a través de nuestro dispositivo móvil Android.

En el presente documento se muestra la realización de un servidor implementado en una Raspberry Pi a través del lenguaje de programación Python, así como una aplicación desplegada a través de Android desde la cual podremos gestionar todo el entorno. Para llevarlo a cabo, se han utilizado los entornos de desarrollo \textit{Visual Studio Code} (Servidor) y \textit{Android Studio} (aplicación).

El resultado de este trabajo se puede encontrar a través del repositorio de GitHub: \href{https://github.com/fmv1001/DomoCamera}{DomoCamera}.
\end{abstract}

\renewcommand*\abstractname{Descriptores}
\begin{abstract}
Android, Servidor, Python, Domótica, Raspberry Pi.
\end{abstract}

\clearpage

% Abstract en inglés
\renewcommand*\abstractname{Abstract}
\begin{abstract}
The boom of home automation and mobile phones has had an impact on society. On the one hand, home automation facilitates a more efficient home management and makes it safer. On the other hand, mobile telephony puts hundreds of technological tools at our disposal to improve people's quality of life. 

This project shows the perfect match between these two sectors, putting them affordable for all, to be able to manage home security from our mobile phone. The project aim is to create an environment that allows us to monitor cameras through our Android mobile device.

This document shows the creation of a server implemented in a Raspberry Pi through the Python programming language, as well as an application deployed through Android from which we can manage the entire environment. To carry it out, the development environments that have been used are \textit{Visual Studio Code} (Server) and \textit{Android Studio} (application).

The result of this work can be found through the GitHub repository: \href{https://github.com/fmv1001/DomoCamera}{DomoCamera}.
\end{abstract}

\renewcommand*\abstractname{Keywords}
\begin{abstract}
Android, Server, Python, Home automation, Raspberry Pi.
\end{abstract}

\clearpage

% Indices
\tableofcontents

\clearpage

\listoffigures

\clearpage

\listoftables
\clearpage

\mainmatter
\capitulo{1}{Introducción}

Hoy en día la tecnologia ha avanzado tanto, que es muy fácil contar con ella a la hora de realizar ciertas tareas, pero cada vez la demandamos 
más para poder trabajar codo con codo con ella, es decir, recibir su ayuda de tal forma que les podamos asignar tareas asegurando que tendrán un
porcentaje de acierto igual o superior al que tendría si lo realizasemos cualquiera de nosotros.

Pero generalmente, para poder llevar a cabo estas tareas, se necesitan dispositivos con una gran cantidad de computo, ya que necesitaremos entrenarlo 
con el objeto u objetos a predecir, siendo está la tarea más importante y la que más capacidad de computo va a necesitar y la que más recursos va consumir.
Tras su entrenamiento, volveremos a consumir recursos para su detección, de tal forma que necesitaremos un equipo lo suficientemente potente para poder realizar
ambas tareas con efectividad y poder obtener buenos resultados.

Debido a esto, el poder entrenar el modelo en un ordenador lo suficientemente potente y seguidamente poder adaptarlo para poder ser utilizado en dispositivos pequeños
como puede ser la Jetson Nano de NVIDIA, y que este dispositivo lo ejecute, sacrificando el porcentaje de acierto pero respetando los tiempos de ejecución, puede facilitar a
muchos trabajadores y/o investigadores en sus trabajos ya que pueden tener una herramienta funcional en poco espacio y además fácil de transportar para poder usarla en diferentes
lugares.

\clearpage

\section{Estructura de la memoria}
La memoria consta de las siguiente estructura:
\begin{list}{\textbullet}{ %
    \addtolength{\itemsep}{-2mm} %
    \setlength{\itemindent}{2mm}}

    \item \textbf{Introducción:} establece el contexto inicial entorno a la idea que se va a desarrollar, además de la estructura del documento y de los materiales que se van a entregar.
    \item \textbf{Objetivos del proyecto:} objtivos que se desean alcanzar durante el desarrollo del proyecto.
    \item \textbf{Conceptos teóricos:} exponer los conceptos que son necesarios disponer para llevar a cabo el proyecto.
    \item \textbf{Técnicas y herramientas:} muestras las técnicas y las herramientas que se han utilizado durante el desarrollo del proyecto.
    \item \textbf{Aspectos relevantes del desarrollo del proyecto:} recopilación de los aspectos más representativos que han tenido lugar durante el desarrollo del proyecto.
    \item \textbf{Trabajos relacionados:} presentación de trabajos que se encuentran relacionados de manera destacable con el desarrollo o el concepto del proyecto.
    \item \textbf{Conclusiones y líneas de trabajo futuras:} descripción de las conclusiones obtenidas durante la realización del proyecto y tras la misma, así como las posibles líneas de mejora.
\end{list}

Además, junto a la presente memoria se incluyen los siguientes anexos relacionados con el desarrollo del modelo de detección y su correspondiente prueba en el dispositivo de Edge Computing:
\begin{list}{\textbullet}{ %
    \addtolength{\itemsep}{-2mm} %
    \setlength{\itemindent}{2mm}}

    \item \textbf{Plan de Proyecto Software:} presentar la planificación temporal llevada a cabo durante el desarrollo del proyecto, así como un estudio de la viabilidad del desarrollo.
    \item \textbf{Especificación de Requisitos:} describir de forma detallada lso objetivos generales y los objetivos del proyecto llevado a cabo.
    \item \textbf{Especificación de diseño:} presentar el diseño final del modelo, describiendo el diseño de datos, procedimental y arquitectónico del desarrollo.
    \item \textbf{Documentación técnica de programación:} en este apartado se describen los conocimientos técnicos más relevantes del proyecto, los cuáles son necesarios para poder continuar con el desarrollo.
    \item \textbf{Documentación de usuario:} apartado dirigido al usuario final, dónde se describen los requisitos necesarios en un dispositivo para poder utilizar la herramienta, la instalación de cada uno de ellos, y un manual de usuario, en el que se mostrarán todas las posibles opciones que dispone la herramienta.
\end{list}
\capitulo{2}{Objetivos del proyecto}

En este apartado se van a presentar los objetivos que han marcado el proyecto, tanto a nivel software, como técnico.

\section{Objetivos Producto Final}
\begin{list}{\textbullet}{ %
    \addtolength{\itemsep}{-2mm} %
    \setlength{\itemindent}{2mm}}

    \item Que el sistema construido sea capaz de entrenar un modelo de detección basado en el algoritmo de YOLO.
    \item Que dicho sistema, sea capaz de convertir el modelo YOLO a uno de Tensorflow, para poder trabajar con él. 
    \item Que sea capaz de detectar objetos a tráves de una imagen o de un vídeo, ya sea uno grabado o la imagen que se captura desde la webcam.
    \item Creación de un script de preprocesado de cara a la evaluación de varias imágenes etiquetadas, de tal forma que podamos obtenerla información original en un único fichero.
    \item Evaluar la calidad de detección del modelo, es decir, calcular el IoU entre las posiciones originales y las predecidas por el modelo, pudiendo obtener su mAP, y resultados sobre la predicción(verdaderos positivos, falsos postivos...), además de que por cada imagen se devolverá la imagen con la posición original, la posición predicha y el IoU.
    \item Mostrar las predicciones en un csv, que muestre el tiempo en el que se ha detectado la predicción, el número de objetos predecidos en dicho instante y el tipo de objeto que es, así como la posición o posiciones en las que se ha encontrado.
    \item Que la apliccación permita contabilizar los objetos que detecte el algorimto
\end{list}

\section{Objetivos Desarrollo}
\begin{list}{\textbullet}{ %
    \addtolength{\itemsep}{-2mm} %
    \setlength{\itemindent}{2mm}}

    \item Convertir el modelo a uno apto para la características de la Raspberry Pi 3.
    \item Usar la plataforma \textit{GitHub} para la organización y gestión del proyecto.
    \item Seguir los principios de la \textit{metodología ágil Scrum}.
    \item Usar herramientas que permitan medir la calidad del código.
    \item Aprender acerca del funcionamiento de los algoritmos de detección de objetos.
    \item Adquirir conocimientos sobre Visión Artificial.
\end{list}


\capitulo{3}{Conceptos teóricos}

Para la compresion de este proyecto, se deben conocer los siguientes conceptos.

\section{Deep Learning} 

El deep learning \cite{deepLearning} es una rama del MachineLearning, donde los algoritmos inspirados en el funcionamiento del cerebro humano (redes neurnales) aprenden a partir de 
grandes cantidades de datos y tratan con un alto número de unidades computacionales..

\section{Edge Computing}

El edge computing \cite{edgeComputing} es un tipo de informática que ocurre, en la ubicación fisica del usuario, en la ubicación de la fuente de los datos o cerca de estas. Permitiendo 
que los usuarios obtengan servicios mas rápidos y fiables.
\capitulo{4}{Técnicas y herramientas}

\section{Metodología}\label{scrum}
A lo largo del proyecto se intentado seguir la \textit{metodología ágil Scrum}, pero adaptada ya que para pdoer aplciar esta metodología es necesario contar con un equipo, en el cuál los diferentes miembros se reparten las roles entre los diferentes miembros que lo conforman. En este caso, lso roles recaeen todo sobre una única persona.

\imagen{scrum}{Pasos de la metodología Scrum}

En primer lugar se encuentra el \textit{Product Backlog} \cite{scrum} que se trata del alcance del proyecto, el cuál va variando dependiendo de los \textit{feedbacks} que se van obteniendo en cada \textit{sprint}.

Seguidamente, se encuentra el \textit{Sprint Backlog}, dónde se marcan los requerimientos que deben de alcanzar durante el \textit{sprint} que se va a iniciar, es decir, se trata de acortar las tareas de cada uno 
de los \textit{sprints}.

La siguiente etapa es el \textit{Sprint}, en la cuál tiene lugar la planificación, la implementación, revisión y retrospectiva de la nueva característica software.
Esta etapa suele tener una duración de una a dos semanas.

Como último paso del proceso, se enecuentra el \textit{incremento del producto}, esta fase consiste en tener una reunión con el cliente con la nueva característica en funcionamiento con el objetivo de obtener una \textit{retroalimentación} por parte del cliente y así volver a empezar el proceso de nuevo.


\section{Lenguaje de programación}
A la hora de empezar un nuevo proyecto es importante relacionado con el \textit{Machine Learning} y el \textit{Edge Computing}, es muy importante seleccionar el lenguaje, con el cuál queremos trabajar destacando dos: \textbf{Python} \cite{python} y \textbf{Matlab} \cite{matlab}.
En este caso se decantó por el uso de \textit{Python}, debido al mayor conocimiento de este lenaguaje y haber trabajado más con este lenguaje que con \textit{Matlab}.
No hay grandes ventajas entre escoger uno u otro.

\subsection*{Python}
Python es un lenguaje de programación multiplataforma y multiparadigma \footnote{Debido a que soporta parcialmente la orientación a objetos, la programación funcional y la imperativa.}, destacando entre sus características la legibilidad y la limpieza del código. Python es un \textit{software open source}, siendo por ello gratuito sus uso.

Fue desarrollado al inicio de la década de los 90 por \textit{Guido van Rosseun}. Fue implementado como el sucesor del lenguaje \textit{ABC} y se encuentra fuertemente influenciado por otros como: ABC, Ada, ALGOL 68, APL, C, C++, CLU, Dylan, Haskell, Icon, Java,
Lips, Modula-3, Perl, Standard ML.

Su principal objetivo es automatizar los procesos, con el fin de minimizar tanto el tiempo de desarrollo, como complicaciones, debido a esto, hoy en día Python es uno de los lenguajes más usados para el desarrolo de todo tipo de aplicaciones.

\section{Algoritmo de detección}
A parte de tener claro el lenguaje que se quiere utilizar, otra característica a tener en cuenta es elegir el \textit{algoritmo de detección} en el que se va a basar el modelo.
Existen diferentes algoritmos:
\begin{list}{\textbullet}{ %
    \addtolength{\itemsep}{-2mm} %
    \setlength{\itemindent}{2mm}}

    \item \textbf{CNN:} \textit{(Convolutional Neuronal Network)} es la opción maás básica que se puede escoger,ya que se parte de una red neuronal convolucional \cite{cnn} la cuál itera la imagen hasta devolver las posiciones de los objetos que detecta.
    
    Está opción trae consigo diferentes inconvenientes:
    \begin{list}{\textbullet}{ %
        \addtolength{\itemsep}{-2mm} %
        \setlength{\itemindent}{2mm}}
        \item Si la imagen detecta varios objetos, situados en zonas opuestas, ¿cuántos píxeles tendremos que desplazarnos en cada dirección?.
        \item El tiempo de cómputo es variable, pudiendo llegar a ser muy largo, ya que por cada movimiento implica una clasificación individual con la red.
        \item Deetectar un objeto dentro de la red, no indica que se poseen los valores 'x' e 'y' de su posición.
        \item Si por un casual el desplazamiento de píxeles que se realiza es muy pequeño, podríamos estar detectadndo el mismo objeto múltiples veces.
        \item Si dos objetos se encuentran muy juntos, se podrían llegara a detectar como un único objeto.    
    \end{list}
    
    \item \textbf{R-CNN:} \textit{(Region Based Convolutional Neural Networks)} surgen en el año 2014, con la siguiente propuesta: determianr primero las regiones de interés de la imagen y después realizar la clasificación de imagenes sobre dichas áreas usando una red preentrenada.\cite{r-cnn}
    Esto implica, que haya un primer algoritmo que detecte las áreas de interés de la imagen, las cuáles pueden ser muchas y de diversos tamaños. Seguidamente, se pasán las diferentes regiones por la CNN, validandose las clases correctas mediante un clasificador bianrio, de tal forma, que se eliminarán
    las que tenga un bajo nivel de confianza. Por último, se ajustaría la posición mediante un regresor.
    
    \imagen{r-cnn-regions}{Pasos de la detección en R-CNN}

    \item \textbf{Fast R-CNN / Faster R-CNN:} Son dos algoritmos que surgen como mejora a R-CNN \cite{faster_rcnn}:
    \begin{list}{\textbullet}{ %
        \addtolength{\itemsep}{-2mm} %
        \setlength{\itemindent}{2mm}}
        \item \textbf{Fast R-CNN:} mejora el algoritmo inicial reutilizando algunso recursos, como las \textit{features} extraídaa por la CNN, de tal forma que se agiliza el entreno y la detección de las imágenes.
        Esta red, posee también mejoras en el cálculo del IoU \textit{(Intersection Over Union)} y en la función de \textit{Loss}. Pero a pesar de esto, no tiene mejoras drásticas en la velocidad de entrenamiento y en la detección.
        \imagen{fast-RCNN}{Arquitectura red Fast-RCNN}
        \clearpage
       \item \textbf{Faster R-CNN:} logra una mejora de velocidad al integrar el algoritmo de \textit{region proposal} \cite{region_proposal} sobre la propia CNN.
       Además aperece el concepto de usar \textit{anchors} fijos, lo cuál consiste en usar tamaños pre calculados para la detección de obejtos de la red.
       \imagen{fasterRCNN}{Arquitectura red Faster-RCNN}
    \end{list}

    \item \textbf{YOLO:} surge en 2016, su nombre viene formado por las siglas de \textit{You Only Look Once} \cite{yolov4}.
    Esta red, como su propio nombre indica hace una única pasada a la red y detecta todos los obejtos para los que ha sido entrenada para clasificar, al realizar un único vistazo obtiene velocidades muy buenas en equipos que no son necesariamente potentes. Lo cuál permite, detecciones en tiempo real de cientos obejtos de forma simúltanea y su ejecución en dispositivos móviles.  
\end{list}
Debido a esto, el modelo escogido ha sido YOLO y en su versión 4, la cuál fue lanzada en Abril del año 2020.

\imagen{resultsYOLOv4}{Comparativa resultados Dataset COCO}

\section{Tensorflow}
\textit{Tensorflow} \cite{tensorflow} es una biblioteca de código abierto, la cuál fue lanzada en el año 2015 por Google, la cuál es muy utilizada por muchas empresas resultado de gran utilidad por su versatilidad y nivel de desarrollo.

Esta herramienta se basa en el \textbf{Deep Learning,} a pesar de que el mercado habia herramientas similares como \textit{DistBelief} \cite{distBelief}, la cuál también fue construida por Google en el año 2011, como un sistema propietario de aprendizaje automático. Su uso creció rápidamente debido al uso de compaañias como \textit{Alphabet}. Pero los años pasaron y el avance de la tecnología 
hizo que las necesidades aumentasen, haciendo que Google invirtiese tiempo para mejorarla, dando lugar a la actual \textbf{Tensorflow,} una herramienta con un conjunto de datos mayor y con mayor capacidad de almacenamiento y modificación.

Se basa en un sistema de redes neuronales, lo cuál permite relacionar varios datos en la red de manera simultánea, es decir, imita lo que hace el cerebro humano.

\section{TensorRT}
\textit{TensorRT} \cite{tensorrt} es un \textit{framework} de aprendizaje automático, el cuál fue publicado por \textbf{Nvidia} para ejecutar inferencias que son interferencias de aprendizaje automático en su hardware. Este \textit{framework}, se encuentra altamente cualificado para ejecutarse en \textit{GPUs Nvidia}, siendo una de las formas más rápidas de ejecutar un modelo en este momento.

\imagen{workflowTensorrt}{Workflow del framework \textit{Tensorrt}}

\section{GitHub}
\textit{GitHub} \cite{github} es una herramienta para el alojamiento del código desarrollado, por cualquier individuo en la web. Esta herramienta permite llevar un control de versiones, siendo esto muy interesante de cara al proyecto que se va a llevar a cabo.
Otra característica destacable es que se puede integrar este sistema en \textit{Visual Studio Code} \cite{visualStudioCode} y que podemos enlazar de forma fácil a nuestro repositorio los códigos realizados con \textit{Google Colab}\cite{colab}.

\clearpage

\section{Herramientas de control de calidad del código}
Las herramientas de control de calidad de código evaluan y procesan de forma automática las líneas de código que conforman el proyecto, comparándolas con unos estándares de calidad. El proyecto será evaluado y dependiendo de si se cumplen o no los estándares y en que medida, la calificación será mejor o peor.

La integración de herramientas de evaluación de código, ayudan a los desarrolladores a evaluar cómo de bueno es un producto software, consiguiendo una mejor mantenibilidad, reducir los errores, hacer revisiones del código centrándose en puntos especificos del código, así como determianr como de buena es la corbertura de pruebas con el objetivo de aumentar el valor del producto software.

Para llevar acabo esta evaluación, se ha contando con la herramienta \textit{SonarCloud}. 

\subsection*{SonarCloud}

SonarCloud \cite{sonar_cloud} es una herramienta de evaluación de código que es capaz, de calcular de forma automática, la duplicación de código, la corbertura del código, su fiabilidad, mantenibilidad, así como su seguridad y sus posibles puntos vulnerables.
Para llevar a cabo sus evaluaciones, SoanrCloud se fija en los estándares y reglas de cada lenguaje.

\begin{list}{\textbullet}{ %
    \addtolength{\itemsep}{-2mm} %
    \setlength{\itemindent}{2mm}}

    \item \textbf{Duplciación de código:} identifica las líneas, fragementos o ficheros de código duplicado.
    \begin{enumerate}
        \item Las duplicaciones en un único archivo, es igual al número de copias encontradase en dicho archivo.
        \item Un archivo se considera duplicado de otro, si el número de líneas iguales supera un determinado umbral.
    \end{enumerate}
    \item \textbf{Seguridad:} para analizar la seguridad de código, se fija en dos puntos: \textit{Security Hostpots} y en las \textit{vulneravilidades}
    \begin{enumerate}
        \item Los Security Hostpots, son partes de código sensible a la seguridad, estos pueden estar bien, pero requieren de una revisión humana.
        \item Las vulnerabilidades de seguridad requieren una acción inmediata, para ello, se fija en las reglas  y en los estándares del lenguaje concreto
    \end{enumerate}
    \item \textbf{Quality Gate:} son un conjunto de condiciones \textit{booleanas}, los cuáles siguen los estándares de vulnerabilidad y revisión de puntos de acceso. Esta métrica, nos indica si el código está listo pra pasar a producción.
\end{list}

\section{Fork}
Fork\ cite{fork}, es un software diseñado para poder realizar de forma gráfica, todas las tareas que se realziarían mediante Git en la consola de comandos.
Es una herramienta multiplataforma, que cuenta con soporte para Windows y MacOS; permite de forma sencilla mantenerse al tanto de repositorios, \textit{branches}, etiquetas, históricos, realizar \textit{commits}... Entre sus características más relevantes se encuentran:
\begin{list}{\textbullet}{ %
    \addtolength{\itemsep}{-2mm} %
    \setlength{\itemindent}{2mm}}

    \item Integración nativa con GitHub Enterprise, GitLab, BitBucket.
    \item Edición y visualización de ramas, merging, histórico de commits.
    \item Simplicidad a la hora de hacer un merge, rebase y push.
    \item Creación, clonación y añadir dispositivos de forma remota.
    \item Muestra la diferencia entre los ficheros con cambios respecto a lo ya cometido.
    \item Soporte de GitFlow, Git Hooks y LFS
\end{list}

\clearpage

\section{LabelImg}
LabelImg \cite{labelImg} es una herramienta gratuita \textit{open source} que permite etiquetar gráficamente imágenes. Se encuentra escrita en Python y de cara a la interfaz gráfica usa QT.
Esta herramienta permite el etiquetado en los froamto de texto \textbf{VOC, XML} o \textbf{YOLO.}
Para llevar acabo el etiquetado de imágenes, lo primero ha realizar es la apertura de la carpeta contenedora de las imágenes (haciendo \textit{click} en \textbf{\textit{Open Dir}}), seguidamente presioanremos la tecla 'w', para comenzar con la seleción de la región a etiquetar, marcaremos el área de la etiqueta y seguidamente nos preguntará el nombre de la clase a la que pertenece dicha etiqueta, creando a su vez un fichero (\textit{classes.txt}) con todas las clases usadas.

\imagen{labelImg}{Funcionamiento de LabelImg}

\section{OIDv4 ToolKit}
OIDv4 ToolKit \cite{OIDv4TK} es un conjunto de herramientas, que permite obtener datos de \textit{train} y de \textit{validation} del \textit{Open Images Dataset V4} 
Para descargar imágenes se usa el siguiente comando:
\begin{verbatim}
    python main.py downloader --classes {nombre-clase} --type_csv {tipo-dato} --limit {número-imágenes}
\end{verbatim}

Dónde \textit{nombre-clase} representa la clase de la cuál se quieren descargar las imágenes(deber ser una clase reconocida por el dataset). 

El campo \textit{tipo-dato} indica si las imágenes van a ser utilizadas para \textit{train} o para \textit{validation}.

Por último, \textit{número-imágenes} indica la cantidad de imágenes que se va a descargar para dicha clase.

Las imágenes descargadas con esta herramienta, ya vienen etiquetadas sólo quedaría convertilas al formato correcto del algoritmo que se vaya a usar.
\capitulo{5}{Aspectos relevantes del desarrollo del proyecto}

En esta sección se van a detallar los aspectos más relevantes que han tenido lugar a lo largo del desarrollo del proyecto. Al tratarse de un desarrolo \textit{software}, y a lo largo de este se han encontrado diferentes retos e inconvenientes, dónde hay que sumarle su posterior ejecución en un \textit{hardware} que posee una pequeña capacidad de conputo. A su vez, se han aplicado buenas prácticas con el objetivo de obtener un resultado de calidad.

\section{Investigación}
El eje principal que posee el proyecto es la \textit{Detección de Objetos}, el cuál a pesar de ser conocido, nunca se había trabajado con él, se puede decir que su conociemnto era puramente teórico. En la misma línea, se desconcoia prácticamente el funcionamiento de la \textit{Raspberry Pi}, la cuál a trabajar de forma internaa con \textit{Raspbian} ha supuesto
una gran ayuda y beneficio al ser un sistema operativo basado en \textit{Debian} el cuál es un sistema que se conocóa de forma previa.
Por todo esto, ha supuesto un doble esfuerzo, especialmente por el primero de ellos, ya que se ha requerido una formación previa par poder conocer el correcto funcionamiento a la hora de entrenar un modelo de detección, así como averiguar cuál era la mejor opción de cara al proyecto.
De cara al objetivo de obtener un mator conocimento acerca de la \textit{Detección de Objetos} se leyeron diferentes \textit{artículos científicos} \cite{Pathak2018,Yang2017, Wu2020}, obteniendo de ellos los suficientes conocimientos sobre su funcionamiento, la calidad de la detección y el redimiento.

\section{Metodología \textit{Scrum}}
Tal y como se comento en el punto \ref{scrum}, el proyecto se ha reliazado siguiendo una metodología ágil, lo cuál nos permite trabajar con \textit{sprints}, de tal manera, que el trabajo a relaizar en cada uno de ellos, se encuentre documentado desde el inicio, para así poder con una mayor eficiencia, pudiendo priorizar las tareas en función de las existentes y contando con diferentes versiones según a la vez que se va siguiendo el desarrollo del proyecto.

Los conocimeientos que se poseían de \textit{Scrum} eran más bien teóricos, pero si que se había tenido la oportunidad de trabajar con ella, durante la realización de las prácticas en empresa.

Una de las principales dificultades encontradas, ha sido la estimación del tiempo que va a ser necesario para pdoer llevarlo a cabo, todo ello apartir de un nombre y descripción, lo cuál ha concluido con resultados no muy exactos, ya que se en algunos casos la estiamción se ha realizado por debajo y en otros por arriba. Si a esto se le suma que ciertas tareas han llevado más tiempo del debido a problemas versiones y de compatibilidades entre sistemas, hay que suamrle que al contar con un determiando número de asiganturas
más la realización de manera paralela de las prácticas, ha conllevado que tareas que de normal llevan pcoo tiempo, han llevado más de lo planeado.

\section{Creación del dataset de entrenamiento y entremiento del modelo de detección}
A lo largo del Grado, a pesar de que se han visto diferentes asignaturas relacionadas con la \textbf{Inteligencia Artificial}, no se ha visto nada relacionado con la \textbf{Visión Artificial}, ni con la \textbf{Detección de Objetos}, siendo esto una desventaja a la hora de crear el dataset del modelo y etiquetarlo con las características correspondientes al algoritmo de detección escogido. Lo que ha conllevado una inversión de horas, de cara a la búsqueda de infromación, así como de las diferentes formas para entrenar el modelo, y con ello la más asequible de cara a requisitos técnicos, debido a que llevar a cabo un entrenamiento de estas características, 
requiere de una tarjeta gráfica potente, así como un equipo los suficientemente potente para llevarlo a cabo, ya que el modelo necesita estar un mínimo de horas entrenando para que el resultado sea medianamente decente y ser funcional.
Debido a esto, las labores de entremiento, se han realizado sobre \textbf{Google Colab}, visto en el Apartado \ref{colab}.

\section{Pruebas realizadas}
A la hora de probar la funcionalidad de modelo que se ha entrenado, se han realizado diferentes pruebas de detección tras su conversión a un modelo \textit{.pb}, el cuál se basa en el grafo de inferencia que calcula \textit{Tensorflow}.
Seguidamente, se comprobó mediante ejecuciones que el modelo es funcional, es decir, que detecta las clases, para las cuáles ha entrenado, detectando imagenes y vídeos.
Con resultados como los que se ven a continuación:

\imagen{coco_detection}{Detección a través del modelo COCO}
\imagen{license_plate_detection}{Detección a través del modelo creado para detectar matrículas}
\imagen{american_plate_detection}{Detección a través del modelo creado para detectar matrículas 2}
\imagen{head_detection}{Detección a través del modelo creado para detectar cabezas}

Tras esto, se realizaron scripts para medir el \textit{accuracy} y para preprocesar las posiciones originales, es decir, convertirlas del formato YOLO al formato \textit{x1, x1, y1, y2}, apto para trabajar con él desde Tensorflow y devolverlas en un único fichero que contenga la ruta de la imagen, el parámetro \textit{x1}, el parámetro \textit{y1}, el parámetro \textit{x2}, el parámetro \textit{y2} y por último el id de la clase que se ha detectado en dicha posición.
Además, por cada imagen que detecta la devuelve representando la posición original, la posición representada por el modelo y el \textit{IoU} entre ambas posiciones.

\clearpage

Tal y como se muestra a continuación junto con los resultados de la evaluación
\imagen{iou2}{Representacioón de la posición original\textit{(verde)}, la posición predecida\textit{(azul)} y el valor del IoU\textit{(amarillo)}.}

\clearpage

Resultados de la evaluación de la calidad del código para el modelo detector de matrículas con un IoU de 0,70:
\begin{figure}[!h]
    \centering
    \includegraphics[width=0.4\textwidth]{mAP_lp}
    \caption{mAP del modelo detector de matrículas}\label{fig:mAP_lp}
\end{figure}
\begin{figure}[!h]
    \centering
    \includegraphics[width=0.4\textwidth]{license_plate_graph}
    \caption{Gráfico \textit{Precission} y \textit{Recall} de la clase \textit{license plate}}\label{fig:license_plate_graph}

\end{figure}
\begin{figure}[!h]
    \centering
    \includegraphics[width=0.4\textwidth]{info_license_plate_predict}
    \caption{Información sobre los \textit{False Predictions} y \textit{True Positives}}\label{fig:info_license_plate_predict}
\end{figure}

\clearpage

Resultados de la evaluación de la calidad del código para el modelo detector de cabezas con un IoU de 0,70:
\begin{figure}[!h]
    \centering
    \includegraphics[width=0.4\textwidth]{evalMAP}
    \caption{mAP del modelo detector de cabezas}\label{fig:evalMAP}
\end{figure}

\begin{figure}[!h]
    \centering
    \includegraphics[width=0.4\textwidth]{prEval}
    \caption{Gráfico \textit{Precission} y \textit{Recall} de la clase \textit{head}}\label{fig:prEval}

\end{figure}

\begin{figure}[!h]
    \centering
    \includegraphics[width=0.4\textwidth]{infoEval}
    \caption{Información sobre los \textit{False Predictions} y \textit{True Positives}}\label{fig:infoEval}
\end{figure}

\clearpage

Resultados de la evaluación de la calidad del código para el modelo COCO:
\begin{figure}[!h]
    \centering
    \includegraphics[width=0.5\textwidth]{cocoMAP}
    \caption{mAP del modelo COCO}\label{fig:cocoMAP}
\end{figure}

\begin{figure}[h!]
    \centering
    \subfloat[Clase \textit{bear}]{
    \label{f:bear}
        \includegraphics[width=0.55\textwidth]{bear.png}}\\
    \subfloat[Clase \textit{tvmonitor}]{
    \label{f:tvmonitor}
        \includegraphics[width=0.55\textwidth]{tvmonitor.png}}\\
    \subfloat[Clase \textit{snowboard}]{
    \label{f:snowboard}
        \includegraphics[width=0.55\textwidth]{snowboard.png}}\\
    \caption{Gráficos Precission y Recall modelo COCO}
    \label{f:graficos precission-recall}
\end{figure}

\begin{figure}[!h]
    \centering
    \includegraphics[width=0.55\textwidth]{cocoPredict}
    \caption{Información sobre los \textit{False Predictions} y \textit{True Positives}}\label{fig:cocoPredict}
\end{figure}

\clearpage

Comparativa de la ejecución de los modelos en Pc y en Raspberry a través de un modelo de Tensorflow Lite
\imagen{compEjecImg}{Comparativa del tiempo de ejecución a través de imágenes}
\imagen{compEjecVid}{Comparativa del tiempo de ejecución a través de vídeos}

Al realizarse estás pruebas bajo el modelo de Tensorflow Lite (ya que es el más apto para las característica de la Raspberry), es muy complicado sacar buenos tiempos en la detección a través de un vídeo y tampoco perder la calidad de detección
del modelo, por ende, se ha ajustado la conversión del modelo YOLO a Tensorflow Lite, de tal forma, que la perdida de la calidad de la detección sea minima.

\clearpage

Comparativa Tensorflow VS Tensorflow Lite (Matrículas)
\imagen{apTF}{Detección en imagen de matrículas (Tensorflow)}
\imagen{apTFL}{Detección en imagen de matrículas (Tensorflow Lite)}

\clearpage

Comparativa Tensorflow VS Tensorflow Lite (Cabezas)
\begin{figure}[!h]
    \centering
    \includegraphics[width=0.6\textwidth]{head1}
    \caption{Detección en imagen de cabezas (Tensorflow)}\label{fig:head1}
\end{figure}
\begin{figure}[!h]
    \centering
    \includegraphics[width=0.6\textwidth]{headTFL}
    \caption{Detección en imagen de cabezas (Tensorflow Lite)}\label{fig:headTFL}
\end{figure}

\clearpage

Comparativa Tensorflow VS Tensorflow Lite (COCO)
\imagen{cocoTF}{Detección en imagen de YOLO (Tensorflow)}
\imagen{cocoTFL}{Detección en imagen de YOLO (Tensorflow Lite)}

\clearpage

\section{Resultados evaluación de código} 
Durante el desarrollo se integró una herramienta que permitiese medir la calidad del código, con el objetivo de controlar la calidad del desarrollo, respecto a los estándares definidos (\textit{ver Apartado }\ref{calidad_codigo}).
En este apartado veremos los resultados que se han ido obteniendo, los cuáles se ejecutaban automáticamente al realizarse un \textit{commit}.

\imagen{code_quality_results}{Resultados de la evaluación de la calidad del código}

A través, del dashboard que posee \textit{SonarCloud} para el proyetco, podemos obtener más información sobre la calidad y vulnerabilidades que posee el código.

\imagen{dashboard_sonar}{Resultados mostrados desde el \textit{Dashboard} de SonarCloud} \label{dashboard}

\imagen{evaluacion_sonar}{Gráfico que muestra la evolución del código}

Gracias al uso de esta herramienta, se ha conseguido reducir los \textit{code smells} hasta el punto de poseer una calificación de A.

En la \textit{Figura} \ref{dashboard} se muestra el dashboard de la herramienta, dónde cada uno de los elementos representa:

\begin{list}{\textbullet}{ %
    \addtolength{\itemsep}{-2mm} %
    \setlength{\itemindent}{2mm}}

    \item \textit{Reliability} Representa la fiabilidad del código, en este aparatdo se encuentran los \textit{bugs} que posee el código.
    \item \textit{Maintainability} Representa la mantenibilidad del código, es decir, el código es fácil de mantener y de modificar. En este apartado, se encuentran los \textit{code smells}, los cuáles son indicaciones de que el código, puede suponer un problema en el futuro, ya que a pesar de que posean esta indicación el código no deja de funcionar.
    \item \textit{Security} Indica el nivel de seguridad que posee el código, a este nivel se encuentran las vulnerabilidades del código, las cuáles indican problemas de seguridad tanto a corto como a largo plazo y deben de ser solventados lo antes posible.
    \item \textit{Security Review} Informan sobre las revisiones de seguridad, es decir, informan sobre posibles vulnerabilidades de seguridad, las cuáles deben de ser revisadas por un humano, para calificar si son o no posibles vulnerabilidades de seguridad.
    \item \textit{Coverage} Indica la cobertura del código con respecto a test realizados, es decir, nos muestra el porcentaje de código que supera los tests.
    \item \textit{Duplication Lines} Muestra el porcentaje de lineas que hay duplicadas a lo largo del código.
\end{list}
\capitulo{6}{Trabajos relacionados}

En esta sección se van a comentar los trabajos relacionados.

A lo largo del proceso previo de investigación, se encontraron diferentes proyectos relacionados con el Object Detection y algunos tambien con Object Detection en dispositivos de Edge Computing.

\begin{list}{\textbullet}{ %
    \addtolength{\itemsep}{-2mm} %
    \setlength{\itemindent}{2mm}}
    \item \textbf{Deep Learning Approaches for Detecting Objects from Images: A Review} \cite{Pathak2018}
    
    En este proyecto se puede ver el procedimeinto de creación de un modelo de detección a través de una red convuluvional, del algortitmo Fast-RCNN y del algoritmo YOLO.
    Así como una comprativa de los resultados obtenidos desde el dataset COCO y desde el dataset de VOC.
    \item \textbf{Deep Learning With Edge Computing: A Review} \cite{Chen2019}
    
    En este trabajo se muestra las ventajas de optar por una herramienta de Edge Computing, sobre no hacerlo, asi como la importancia de la \textit{latecnia}, de la \textit{escalabilidad} y de la \textit{privacidad}, ya que estás características hacen del Edge Computing una opción viable de cara al control de estos tres puntos.

    \item \textbf{Making accurate object detection at the edge: review and new approach} \cite{Huang2021}
    
    A lo largo de este trabajo se muestra las diferentes medidas de \textit{accuracy} de diferentes modelos de detección a través de su comparativa de tiempo en una Raspberry Pi 3B+ y en una 4B+. A su vez, a lo largo del trabajo se diseña una nueva red neuronal la CNN-RIS, la cuál posee la arquitectura, que se muestra a continuación y usa el algoritmo de entrenamiento que se muestra seguidamente, con la cuál consigue mejorar los resultados obtenidos de los otros modelos tanto en velocidad como en \textit{accuracy}(concretamente en la Raspberry PI 4B+)
    \imagen{arq_cnn-ris}{Arquitectura Red CNN-RIS} 
    \imagen{arq_cnn-ris_table}{Contenido Arquitectura Red CNN-RIS} 
\end{list}
\capitulo{7}{Conclusiones y Líneas de trabajo futuras}

En este apartado se van a agrupar tanto las conclusiones a las que se llega tras el desarrollo del proyecto, como los posibles puntos de mejora, como las líneas de trabajo futuras.

\section{Conclusiones}
Conclusiones a las que se llega tras el posterior desarrollo del proyecto.
\begin{itemize}
    \item Los objetivos del proyecto se han cumplido satisfactoriamente
    \begin{list}{\textbullet}{ %
        \addtolength{\itemsep}{-2mm} %
        \setlength{\itemindent}{2mm}}
        \item \texttt{Object Detection}. Posee un algorimto para poder entrenar cualquier modelo con el algoritmo YOLO en la versión 4. Así como, un script para convertir dicho modelo a uno apto para Tensorflow, y así poder detectar vídeos, imágenes y/o trabajar con Obejct Tracking.
        A su vez, podemos exportar las posiciones de cada uno de estos scripts a un fichero \textit{csv}, como guardar el resultado deseado en un vídeo o en una imagen.
        Pero sobre todo, nos permite medir la caldiad del modelo entrenado, con el objetivo de evaluar su calidad de evalaución.  
    \end{list}
    \item La lectura de artículos científicos ha sido algo prácticamente nuevo, siendo necesarias numerosas horas para su interpretación, se debe destacar que según se avanzaba en el desarrollo del proyecto, la familiarización con éstos ha sido satisfactoria, permitiendo asimilar la información como la idea que se iba a implementar.
    \item El trabajar con Object Detection, a través del algoritmo de detección de YOLO, permite que sea rápido y su calidad de detección sea bastante buena.
    \item La creación de un modelo de detección, así como su respectivo dataset de entremiento ha permitido aumentar el conocmiento que se tenia sobre la detección de objetos, así cmo trabajar con herramientas que al inicio del proyecto eran totalmente desconcoidas. 
\end{itemize}

Como \textbf{conclusión} cabe destacar, la oportunidad tanto de aprendizaje como la de trabajar en un proyecto nuevo ha supuesto. Desde el inicio ha sido necesario dominar ciertas tecnologías vistas (y nuevas) a lo largo de todo el Grado, incluso a la hora de escribir la documentación.

\section{Líneas de trabajo futuras}
Dadas las características del proyecto, existen diferentes líneas de mejora y trabajo futuras.

\begin{list}{\textbullet}{ %
    \addtolength{\itemsep}{-2mm} %
    \setlength{\itemindent}{2mm}}
    \item Crear scripts de entrenamiento, conversión y adaptación de los scripts de detección a YOLOv5.
    \item Permitir la importación y conversión a otros algoritmos diferentes de YOLO y sus versiones.
    \item Crear funciones que permitan apoyar la detección de los objetos, y en función de la clase o clases que se vayan a detectar se realize unas u otros, por ejemplo, en el detector de matrículas que permita leerlas y exportalas a un fichero, en modelos que permitan detectar vehículos contar vehículos en ambas direcciones, con el objetivo de contar los que entran y salen de una carretera y exportarlos a un fichero \textit{csv}.
    \item Realizar un proceso de refactorización código, ya que si el proyecto crece y por ende, aumentan sus funcionalidades.
    \item Tras la adaptaciones a otros algoritmos de detección, adaptarlo en función de cada algortimo, con el objetivo de tratarlo como una librería y subirlo a PIP \cite{pip}.
\end{list}


\bibliographystyle{IEEEtran}
\bibliography{IEEEabrv,bibliografia}

\end{document}