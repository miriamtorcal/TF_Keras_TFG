\apendice{Plan de Proyecto Software}

\section{Introducción}
En este apéndice se va a mostrar la planificación del proyecto, la cuál es la base sobre la crea el proyecto \textit{software}. Desde el punto de vista de la temporalidad y viabilidad. 
Siendo está una parte fundamental del proyecto, ya que permite visualizar el escenario en el que se desarrollará, de tal forma que podamos realizar una alineación estrategica de los elementos que deben de ser completados, con el objetivo de finalizarlo correctamente.

\section{Planificación temporal}
La planificación temporal se

\section{Estudio de viabilidad}
En esta sección se va a desarrollar el estudio de la viabilidad del proyecto, con el objetivo de disponer de una visión global de los posibles benficios en contraposición del csote que supone el desarrollo del proyecto.
El desarrolo de cualquier proyecto \textit{software} lleva consigo una serie de riesgos, entre los que destacan la experiencia del equipo desarrollador, el tamaño del proyecto, el tiempo del que se dispone para llevarlo a cabo..., influyendo todo ello en el resultado final del proyecto.

\subsection{Viabilidad económica}
El primer paso del estudio, consiste en calcular la viabilidad económica del proyetco, para ello se deben de reportar y analizar los costes/beneficios que habría supuesto el proyecto en España \footnote{Nos referimos a España ya que, es el país dónde tiene lugar el desarrollo, y por ello se tendrán en cuenta los valores económica de dicho país.}.

\subsubsection{Costes}
El llevar a acabo la realización de un proyecto de esta envergadura, posee una serie de costes, tanto fijos cómo variables, los cuáles se van a desglosar en \textit{hardware}, \textit{software}, \textit{personal}\dots

\textbf{Costes Hardware}
Para llevar a cabo el proyecto, es necesario contar con algún que otro sistema \textit{hardware}, el primero de ellos es un equipo portátil. Se utilizará un MSI GS63 Stealth 8RE, con un procesador Intel Core i7 (8ª generación) de 6 núcleos a 2.2 GHz, con 16 GB de memoria RAM y una tarjeta gráfica 1060 de 6 GB, el cuál posee un precio del mercado de 2200 €, se tiene en cuenta que el tiempo de vida útil del equipo se enceuntra en torno a los seis años, 
por ello, de cara a los cálculos se tendrá en cuenta el tiempo de vida útil del inmovilizado, es decir, tres años.

A su vez, se necesita una Raspberry Pi, para poder trabajar con la aplicación desde un dispositivo con una capacdad de computo muy inferior al equipo principal de trabajo, en este caso se contará con una Raspberry Pi 3B, la cuál cuenta con un procesador Broadcom BCM2837 con 4 núcleos a 1,2 GHz y 1 GB de memoria RAM, la cuál posee un precio de 38 € y teniendo en cuenta que la vida útil es de aproximádamente seis años.
De la misma forma que con el portátil, se usará el tiempo medio de inmoviliado, es decir, tres años.

La amortización para ambos equipos es distinta, pero el tiempo de uso es el mismo, de noviembre de 2021 a septiembre de 2022, ambos incluidos, por ende, diez meses. A continuación, se detallarán todos los costes hardware (Tabla \ref{tab:costesHardw})

\begin{table}[H]
    \centering
    \begin{tabular}{lrr}
        \toprule
        \textbf{Concepto} & \textbf{Coste (\officialeuro)} & \textbf{Coste Amortizado (\officialeuro)}\\
        \midrule
        MSI GS63 Stealth 8RE & 2.200 & 611,11\\
        Raspberry Pi 3B & 38 & 10,55 \\
        \midrule
        \textbf{Total} & 2.238 & 621.66 \\
        \bottomrule
    \end{tabular}
    \caption{Costes de \emph{hardware}.}\label{tab:costesHardw}
\end{table}

\textbf{Costes Software}\\
Para el desarrollo del proyecto y uso del \textit{software} necesario, se necesita la adquisición de determinadas licencias, las cuáles se muestran a continuación , con su amortización correspondiente (Tabla \ref{tab:costesSoft}).

\begin{table}[H]
    \centering
    \begin{tabular}{lrr}
        \toprule
        \textbf{Concepto} & \textbf{Coste (\officialeuro)} & \textbf{Coste Amortizado (\officialeuro)}\\
        \midrule
        Fork & 49.99 & 41,66\\
        \TeX Works & 0 & 0 \\
        Windows 10 Pro & 259 & 215 \\
        Raspbian & 0 & 0 \\
        Visual Studio Code & 0 & 0 \\
        SonarCloud & 120 & 100 \\
        TensorFlow & 0 & 0 \\
        \midrule
        \textbf{Total} & 428,99 & 356.66 \\
        \bottomrule
    \end{tabular}
    \caption{Costes de \emph{software}.}\label{tab:costesSoft}
\end{table}

\textbf{Coste de personal}\\
El desarrollo del proyecto se ha llevado a cabo por un desarrollador y el tutor del proyecto.

\begin{itemize}
    \item El salario del desarrollador se calcula según~\cite{sueldoJunior}, siendo el salario medio anual del 20.201~\officialeuro{} netos al año.
    \item El salario del tutor se calcula según~\cite{sueldoInvest}, siendo el salario medio anual de 22.784~\officialeuro{} netos al año. Con una carga de trabajo de dos horas semanales.
\end{itemize}

La duración total del proyecto es de 40 semanas, por lo tanto el tutor trabajará un total de 80 horas, mientras que el desarrollador trabajará 800.

El IRPF es del 24\% y la retribución a la Seguridad Social, calculada tal cual se plantea en~\cite{ssCotizacion} por el Ministerio de Inclusión, Seguridad Social y Migraciones; se contribuye con un 31,40\% en total. Estando dividido en:
\begin{itemize}
    \tightlist
    \item 23,60\% de contingencias comunes~\cite{BOEPCM2442022}.
    \item 5,50\% por desempleo de tipo general~\cite{BOEPCM2442022}.
    \item 0,20\% destinado al Fondo de Garantía Salarial~\cite{BOEPCM2442022}.
    \item 0,60\% de formación profesional~\cite{BOEPCM2442022}.
    \item 1,50\% de tipo de cotización por accidentes de trabajo y enfermedades profesionales~\cite{BOEENFERMEDADES}.
\end{itemize}

\begin{table}[H]
    \centering
    \begin{tabular}{lrr}
        \toprule
        \textbf{Concepto} & \textbf{Desarrollador (\officialeuro)} & \textbf{Tutor (\officialeuro)}\\
        \midrule
        Salario total neto & 1.082,20 & 81 \\
        Retención IRPF & 259,73 & 19,44 \\
        Seguridad Social & 339,81 & 25,43 \\
        \midrule
        \textbf{Total salario bruto} &  1.681,74 & 125,87  \\
        \midrule
        \textbf{Total 10 meses} & 16.817,4 & 1.258,70 \\
        \bottomrule
    \end{tabular}
    \caption{Costes de personal.}\label{tab:costesPersonal}
\end{table}

En la Tabla \ref{tab:costesPersonal} se muestran de forma desgloasada los costes del personal, dando como coste final del personal:

16.817,40 \officialeuro (Desarrollador) + 1.258,70 \officialeuro (Tutor) = 18.076,10 \officialeuro

\textbf{Otros Costes}\\
Aqui se indican, costes que no son agrupables en ningún de los otros grupos, pero hay que tenerlos en cuenta.
Ver Tabla \ref{tab:costesOtros}

\begin{table}[H]
    \centering
    \begin{tabular}{lr}
        \toprule
        \textbf{Concepto (\officialeuro)} & \textbf{Coste (\officialeuro)} \\
        \midrule
        Logo & 20 \\
        Memoria impresa & 250 \\
        Alquiler oficina & 1600 \\
        Internet & 136\\
        Electricidad & 140 \\
        Agua & 51\\
        Calefacción & 230 \\
        \midrule
        \textbf{Total} & 2.427 \\
        \bottomrule
    \end{tabular}
\caption{Otros costes.}\label{tab:costesOtros}
\end{table}

A continaución, ver Tabla~\ref{tab:costesTotal}, se muestran los gastos totales del proyecto.

\begin{table}[H]
    \centering
    \begin{tabular}{lr}
        \toprule
        \textbf{Categoría (\officialeuro)} & \textbf{Coste (\officialeuro)} \\
        \midrule
        \textit{Hardware} & 2.238 \\
        \textit{Software} & 428,99 \\
        Personal & 18.076,10 \\
        Otros & 2.427 \\
        \midrule
        \textbf{Total} & 23.170,09 \\
        \bottomrule
    \end{tabular}
    \caption{Costes totales.}\label{tab:costesTotal}
\end{table}

\textbf{Beneficios}\\
De cara a la comercialización del proyecto, si se deseease obtener un beneficio económico del proyecto, se podría de manera sencilla crear un acceso por usuarios y que estos tengan diferentes niveles para acceder, número máximo de detecciones, número máximo de modelos que permite subir.
A su vez, también se podría plantear cómo un sistema de subscripción. En la Tabla \ref{tab:opcBen}

\begin{table}[H]
    \centering
    \begin{tabular}{lcrr}
        \toprule
        \textbf{Tipo} & \textbf{Detecciones} & \textbf{Modelos} & \textbf{Precio (\officialeuro)} \\
        \midrule
        \textit{Trial} & 5 & 2 & 0 \\
        \textit{Estudiante} & 25 & 5 & 10 \\
        \textit{Investigador} & 120 & 45 & 40 \\
        \textit{Equipo} & 300 & 100 & 150 \\
        \textit{Empresa} & Ilimitadas & 950 & 2000 \\
        \bottomrule
    \end{tabular}
    \caption{Posibles opciones para obtener beneficio del proyecto.}\label{tab:opcBen}
\end{table}

\textbf{Conclusiones}
Analizando los beneficios que se reportan en contraposición a los costes que conlleva el desarrollo del proyecto, queda demostrado su viabilidad. Se tiene ene cuenta, que el proyecto, no tiene la necesidad de ser mantenido, salvo por tener que añadir nuevos algoritmos de detección (y por ende, aceptar sus extensiones).


\subsection{Viabilidad legal}
En esta sección, principalmente se va a discutir el tema de la licencias, ya que al tratarse de un producto \textit{software} es la temática principal.

Según \cite{softwareLicense}, una licencia de \textit{software} es un contrato entre la entidad que ha creado y suministardo una aplicación, el código fuente de está o un producto relacionado con el mismo, y el usuario o usuarios finales.
La licencia es un documento, generalmente de texto, el cuál ha sido diseñado para proteger la propiedad intelectual del desarrollador del \textit{software} y para evitar reclamos en su contra debido a su uso.

Está, a su vez, proporciona, definiciones legalmente vinculantes para la definición y el uso del \textit{software}. Los derechos que posee el usuario final, también se encuentran detallados en la licencia.

\clearpage
\subsection{\textit{Software}}
La licencia más importante, es sobre la cuál se encuentra el \textit{software} desarrollado permitiendo así su uso.

El proyecto se encuentra licenciado bajo MIT. Ver Figura \ref{fig:license}.
\imagenflotante{license}{license}