\capitulo{2}{Objetivos del proyecto}

En este apartado se van a presentar los objetivos que han marcado el proyecto, tanto a nivel software, como técnico.

\section{Objetivos Producto Final}
\begin{list}{\textbullet}{ %
    \addtolength{\itemsep}{-2mm} %
    \setlength{\itemindent}{2mm}}

    \item Creación del script que permita entrenar el modelo de detección con las clases seleccionadas, obteniendo como resltado el modelo entrenado (al realizarse mediante YOLO, devolverá un fichero .weights)
    \item Convertir el modelo YOLO (.weights) a un modelo Tensorflow (.pb) para poder trabajar con él.
    \item Creación de los scripts que permitan la detección de los objetos, ya sea partiendo de una imagen o de vídeo (cargando un vídeo o conectando la webcam), los cuáles contarán con diferentes flags de accion durante la ejecución.
    \item Creación de un script de preprocesado de cara a la evaluación de varias imágenes etiquetadas, de tal forma que podamos obtenerla información original en un único fichero.
    \item Medir la calidad de la evaluación del modelo, es decir, calcular el IoU entre las posiciones originales y las predecidas por el modelo, pudiendo obtener su mAP, y resultados sobre la predicción(verdaderos positivos, falsos postivos...), además de que por cada imagen se devolverá la imagen con la posición original, la posición predicha y el IoU.
    \item Mostrar las predicciones en un csv, que muestre el tiempo en el que se ha detectado la predicción, el número de objetos predecidos en dicho instante y el tipo de objeto que es, así como la posición o posiciones en las que se ha encontrado.
    \item Crear un script, que permita identificr las clases con un tracker, es decir, que identifique las clases y nos las vaya etiquetando según vaya detectando.
\end{list}

\section{Objetivos Desarrollo}
\begin{list}{\textbullet}{ %
    \addtolength{\itemsep}{-2mm} %
    \setlength{\itemindent}{2mm}}

    \item Convertir el modelo a uno apto para la características de la Jetson Nano.
    \item Usar la plataforma \textit{GitHub} para la organización y gestión del proyecto.
    \item Seguir los principios de la \textit{metodología ágil Scrum}.
    \item Usar herramientas que permitan medir la calidad del código.
\end{list}

