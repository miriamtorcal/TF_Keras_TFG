\apendice{Especificación de Requisitos}

\section{Introducción}
En este apéndice se recogen las necesidades funcionales que deberán de ser soportadas por el sistema que va a ser desarrollado. Con el objetivo de obtener una buena documentación, deben de identificarse y describirse los requesitos que tienen que ser satisfacidos por el sistema, pero sin entrar en su proceso de realización.

\section{Objetivos generales}
Los objetivos del proyecto son los siguientes:
\begin{enumerate}
    \item Realización de una aplicación que permita la detección de objetos en diferentes tipos de represenatciones visuales y de diferentes formas.
    \item Creación de los \textit{datasets}, con los que se entrenarán los modelos y la obtención de estos en el formato, del algoritmo de detección usado.
    \item Creación/mejora de los diferentes scripts que permiten tanto la detección de los objetos, la conversión de formato de los modelos, la evaluación de \textit{accuracy} de lso diferentes modelos entrenados. 
    \item Conseguir que las interfaces a diseñar sean intuitivas y fáciles de utilizar. Deberán de ser transparentes al usuario, de tal forma, que se poseea un manejo de errores, ya sea producidos a nivel interno o por el usuario durante su uso.
    \item Permitir el uso de diferentes modelos asociados a la librería de Tensorflow.   
\end{enumerate}

\section{Catalogo de requisitos}
En esta sección se van a definir de forma clara y precisa todas las funcionalidades y restricciones del sistema.

\subsection{Requisitos funcionales}
\begin{itemize}
    \tightlist
    \item \textbf{RF-1 Uso de YOLOv4 como algoritmo de aprendizáje.} Se debe de ser capaz de entrenar un modelo mediante el algoritmo de detección de YOLOv4 y posteriormente utilizar dicho modelo para predecir sobre una representación visual.
    \begin{itemize}
    \tightlist
    \item \textbf{RF-1.1 Entrenar el modelo.} Se debe de poder entrenar tantos modelos como desee el usuario, los cuáles llevarán un periodo indefinido de tiempo, obteniendose en función de este mejores o peores resultados.
    \item \textbf{RF-1.2 Descarga del modelo.} Se debe de poder descargar/guardar el modelo, de cara a poder utilizarlo posteriormente.
    \item \textbf{RF-1.3 Conversion del modelo.} Se debe de poder convertir el modelo entrenado, el cuál posee un formato \textit{weights} a uno formato apto de la librería TensorFlow.
    \end{itemize}
    \item \textbf{RF-2 Detección de objetos.}
    \begin{itemize}
      \tightlist
      \item \textbf{RF-2.1 Detección de objetos sobre una imagen.} Permitir el reconocimiento de los objetos que posee el modelo, a través de una imagen, guardar el resultado de la detección en un vídeo y un fichero CSV con las posiciones.
      \item \textbf{RF-2.2 Detección de objetos sobre un vídeo.} Permitir el reconocimiento de los objetos que posee el modelo, a través de una vídeo, guardar el resultado de la detección en un vídeo y un fichero CSV con las posiciones.
      \item \textbf{RF-2.3 Detección de objetos visualizados por la webcam.} Permitir el reconocimiento de los objetos que posee el modelo, a través de el vídeo que captura la webcam, guardar el resultado de la detección en un vídeo y un fichero CSV con las posiciones
      \item \textbf{RF-2.4 Contabilización de objetos sobre un vídeo.} Permitir la contabilización de los objetos que posee el modelo, a través de un vídeo, , guardar el resultado de la contabilización en un vídeo y un fichero CSV con las posiciones.
      \item \textbf{RF-2.5 Deteccion de objetos sobre una imagen web, a través de su URL.} Permitir el reconociemiento de los objetos que posee el modelo, a través de la lectura de la URL de la imagen, , guardar el resultado de la contabilización en un vídeo y un fichero CSV con las posiciones.
      \item \textbf{RF-2.6 Detección de objetos sobre un vídeo que se encuentra en YouTube.} Permitir la contabilización de los objetos que posee el modelo, a través de la lectura y procesamiento del vídeo de YouTube, guardar el resultado de la contabilización en un vídeo y un fichero CSV con las posiciones.  
    \end{itemize}
    \item \textbf{RF-3 Evaluación del modelo.}
    \begin{itemize}
      \item \textbf{RF-3.1 Preprocesamiento de la información.} El usuario debe ser capaz de obtener a partir de un dataset acto, un fichero que contenga la información necesaria para su evaluación.
      \item \textbf{RF-3.2 Algoritmo de evaluación del modelo.} El usuario debe ser capaz de poder evaluar el modelo que desee a partir del fichero que obtuvo previamente.
      \item \textbf{RF-3.3 Representación de los resultados.} El usaurio debe ser capaz de entender el resultado de la evaluación del algoritmo.
    \end{itemize}
\end{itemize}

\subsection{Requisitos no funcionales}
\begin{itemize}
\item \textbf{RNF-1 Usabilidad.} La plataforma debe de ser fácil tanto de aprender a utilizar como clara a la hora de reportar los errores que se puedan cometer. La interfaz debe ser intuitiva.
\item \textbf{RNF-2 Rendimiento.} La interfaz web debe de tener unos tiempos de carga razonables.
\item \textbf{RNF-3 Escalabilidad.} La plataforma debe soportar que se le añadan nuevas funcionalidades con relativa facilidad.
\item \textbf{RNF-4 Disponibilidad.} La plataforma debe de ser accesible a través de Internet sin importar la geolocalización del cliente.
\item \textbf{RNF-5 Mantenibilidad.} La plataforma debe cumplir los estándares de código de cada uno de los lenguajes en los que se desarrolla. 
\item \textbf{RNF-6 Soporte.} La plataforma debe dar soporte a ficheros CSV, MP4, JPG como mínimo. Así como ser compatible con HTML5.
\end{itemize}

\section{Especificación de requisitos}
Dentro de esta sección de va a desarrollar el Diagrama de Casos de Uso, ver Figura \ref{fig:useCases} y su explicación correspondiente.

\imagen{useDiagram}{Diagrama de Casos de Uso} \label{fig:useCases}

\subsection{Casos de Uso}
A continuación se adjutan las tablas de los casos de uso anteriormente planteados.

% Caso de Uso 1 -> Crear modelo
\begin{table}[p]
  \centering
  \begin{tabularx}{\linewidth}{ p{0.21\columnwidth} p{0.71\columnwidth} }
  \toprule
  \textbf{CU-1} & \textbf{Crear modelo}\\
  \toprule
  \textbf{Versión} & 1.0\\
  \textbf{Autor} & Miriam Torres Calvo\\
  \textbf{Requisitos asociados} & RF-1.1, RF-1.2\\
  \textbf{Descripción} & Permite al usuario obtener un modelo entrenado.\\
  \textbf{Precondición} & Tiene que existir un dataset previo con todas las clases que se desea detectar.\\
  \textbf{Acciones} &
  \begin{enumerate}
  \def\labelenumi{\arabic{enumi}.}
  \tightlist
  \item El usuario genera el dataset.
  \item Ejecuta el script de entrenamiento.
  \item Obtiene el modelo entrenado en el formato de YOLOv4.
  \end{enumerate}\\
  \textbf{Postcondición} & \\
  \textbf{Excepciones} & No existen excepciones posibles.\\
  \textbf{Importancia} & Alta \\
  \bottomrule
  \end{tabularx}
  \caption{CU-1 Crear modelo.}
\end{table}

% Caso de Uso 2 -> Conversión del modelo
\begin{table}[p]
  \centering
  \begin{tabularx}{\linewidth}{ p{0.21\columnwidth} p{0.71\columnwidth} }
  \toprule
  \textbf{CU-1} & \textbf{Convertir modelo}\\
  \toprule
  \textbf{Versión} & 1.0\\
  \textbf{Autor} & Miriam Torres Calvo\\
  \textbf{Requisitos asociados} & RF-1.3\\
  \textbf{Descripción} & Permite al usuario convertir el modelo entrenado a un formato de TensorFlow.\\
  \textbf{Precondición} & Tiene que existir un modelo entrenado en formato \textit{weights}.\\
  \textbf{Acciones} &
  \begin{enumerate}
  \def\labelenumi{\arabic{enumi}.}
  \tightlist
  \item El usuario escoge el modelo que desea convertir.
  \item El usuario escoge el fichero de etiquetas que se usará durante la conversión.
  \item Escoge el el modelo de Tensorflow al que desea convertir.
  \item Ejecuta el script correspondiente.
  \item El usuario obtiene el modelo convertido.
  \end{enumerate}\\
  \textbf{Postcondición} & \\
  \textbf{Excepciones} & Si el número de clases del modelo es mayor que las qu eposee el fichero de etiquetas.\\
  \textbf{Importancia} & Alta \\
  \bottomrule
  \end{tabularx}
  \caption{CU-2 Convertir modelo.}
\end{table}

% Caso de Uso 3 -> Crear modelo
\begin{table}[p]
  \centering
  \begin{tabularx}{\linewidth}{ p{0.21\columnwidth} p{0.71\columnwidth} }
  \toprule
  \textbf{CU-1} & \textbf{Evaluar modelo}\\
  \toprule
  \textbf{Versión} & 1.0\\
  \textbf{Autor} & Miriam Torres Calvo\\
  \textbf{Requisitos asociados} & RF-3.1, RF-3.2, RF-3.3\\
  \textbf{Descripción} & Permite al usuario obtener un modelo entrenado.\\
  \textbf{Precondición} & Tiene que existir un dataset previo con todas las clases que se desea detectar.\\
  \textbf{Acciones} &
  \begin{enumerate}
  \def\labelenumi{\arabic{enumi}.}
  \tightlist
  \item El usuario genera el dataset.
  \item Ejecuta el script de entrenamiento.
  \item Obtiene el modelo entrenado en el formato de YOLOv4.
  \end{enumerate}\\
  \textbf{Postcondición} & \\
  \textbf{Excepciones} & Si el número de clases del modelo es mayor que las qu eposee el fichero de etiquetas.\\
  \textbf{Importancia} & Alta \\
  \bottomrule
  \end{tabularx}
  \caption{CU-3 Evaluar modelo.}
\end{table}

% Caso de Uso 4 -> Detectar modelo
\begin{table}[p]
  \centering
  \begin{tabularx}{\linewidth}{ p{0.21\columnwidth} p{0.71\columnwidth} }
  \toprule
  \textbf{CU-1} & \textbf{Detectar objetos}\\
  \toprule
  \textbf{Versión} & 1.0\\
  \textbf{Autor} & Miriam Torres Calvo\\
  \textbf{Requisitos asociados} & RF-2.1, RF-2.2, RF-2.3, RF-2.4, RF-2.5, RF-2.6\\
  \textbf{Descripción} & Permite al usuario obtener un modelo entrenado.\\
  \textbf{Precondición} & Tiene que existir un dataset previo con todas las clases que se desea detectar.\\
  \textbf{Acciones} &
  \begin{enumerate}
  \def\labelenumi{\arabic{enumi}.}
  \tightlist
  \item El usuario genera el dataset.
  \item Ejecuta el script de entrenamiento.
  \item Obtiene el modelo entrenado en el formato de YOLOv4.
  \end{enumerate}\\
  \textbf{Postcondición} & \\
  \textbf{Excepciones} & Si el número de clases del modelo es mayor que las que posee el fichero de etiquetas. Extensión del fichero inválida\\
  \textbf{Importancia} & Alta \\
  \bottomrule
  \end{tabularx}
  \caption{CU-4 Detectar modelo.}
\end{table}