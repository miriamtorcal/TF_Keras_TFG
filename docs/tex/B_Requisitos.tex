\apendice{Especificación de Requisitos}

\section{Introducción}
En este apéndice se recogen las necesidades funcionales que deberán de ser soportadas por el sistema que va a ser desarrollado. Con el objetivo de obtener una buena documentación, deben de identificarse y describirse los requesitos que tienen que ser satisfacidos por el sistema, pero sin entrar en su proceso de realización.

\section{Objetivos generales}
Los objetivos del proyecto son los siguientes:
\begin{enumerate}
    \item Realización de una aplicación que permita la detección de objetos en diferentes tipos de represenatciones visuales y de diferentes formas.
    \item Creación de los \textit{datasets}, con los que se entrenarán los modelos y la obtención de estos en el formato, del algoritmo de detección usado.
    \item Creación/mejora de los diferentes scripts que permiten tanto la detección de los objetos, la conversión de formato de los modelos, la evaluación de \textit{accuracy} de lso diferentes modelos entrenados. 
    \item Conseguir que las interfaces a diseñar sean intuitivas y fáciles de utilizar. Deberán de ser transparentes al usuario, de tal forma, que se poseea un manejo de errores, ya sea producidos a nivel interno o por el usuario durante su uso.
    \item Permitir el uso de diferentes modelos asociados a la librería de Tensorflow.   
\end{enumerate}

\section{Catalogo de requisitos}
En esta sección se van a definir de forma clara y precisa todas las funcionalidades y restricciones del sistema.

\subsection{Requisitos funcionales}
\begin{itemize}
    \tightlist
    \item
      \textbf{RF-1 Uso de algoritmos de aprendizaje automático.} Se debe de ser capaz de entrenar un modelo mediante el algoritmo de detección de YOLOv4 y posteriormente utilizar dicho modelo para predecir sobre una representación visual.
    
    \begin{itemize}
    \tightlist
    \item \textbf{RF-1.1 Entrenar el modelo.} Se debe de poder entrenar tantos modelos como desee el usuario, los cuáles llevarán un periodo indefinido de tiempo, obteniendose en función de este mejores o peores resultados.
    \item \textbf{RF-1.2 Descarga del modelo.} Se debe de poder descargar/guardar el modelo, de cara a poder utilizarlo posteriormente.
    \item \textbf{RF-1.3 Conversion del modelo.} Se debe de poder convertir el modelo entrenado, el cuál posee un formato \textit{weights} a uno formato apto de la librería TensorFlow.
    \end{itemize}
\end{itemize}

\section{Especificación de requisitos}


