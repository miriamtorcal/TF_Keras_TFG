\apendice{Especificación de Requisitos}

\section{Introducción}
En este apéndice se recogen las necesidades funcionales que deberán de ser soportadas por el sistema que va a ser desarrollado. Con el objetivo de obtener una buena documentación, deben de identificarse y describirse los requesitos que tienen que ser satisfacidos por el sistema, pero sin entrar en su proceso de realización.

\section{Objetivos generales}
Los objetivos del proyecto son los siguientes:
\begin{enumerate}
    \item Realización de una aplicación que permita la detección de objetos en diferentes tipos de represenatciones visuales y de diferentes formas.
    \item Creación de los \textit{datasets}, con los que se entrenarán los modelos y la obtención de estos en el formato, del algoritmo de detección usado.
    \item Creación/mejora de los diferentes scripts que permiten tanto la detección de los objetos, la conversión de formato de los modelos, la evaluación de \textit{accuracy} de lso diferentes modelos entrenados. 
    \item Conseguir que las interfaces a diseñar sean intuitivas y fáciles de utilizar. Deberán de ser transparentes al usuario, de tal forma, que se poseea un manejo de errores, ya sea producidos a nivel interno o por el usuario durante su uso.
    \item Permitir el uso de diferentes modelos asociados a la librería de Tensorflow.   
\end{enumerate}

\section{Catalogo de requisitos}
En esta sección se van a definir de forma clara y precisa todas las funcionalidades y restricciones del sistema.

\subsection{Requisitos funcionales}
\begin{itemize}
    \tightlist
    \item \textbf{RF-1 Uso de YOLOv4 como algoritmo de aprendizáje.} Se debe de ser capaz de entrenar un modelo mediante el algoritmo de detección de YOLOv4 y posteriormente utilizar dicho modelo para predecir sobre una representación visual.
    \begin{itemize}
    \tightlist
    \item \textbf{RF-1.1 Entrenar el modelo.} Se debe de poder entrenar tantos modelos como desee el usuario, los cuáles llevarán un periodo indefinido de tiempo, obteniendose en función de este mejores o peores resultados.
    \item \textbf{RF-1.2 Descarga del modelo.} Se debe de poder descargar/guardar el modelo, de cara a poder utilizarlo posteriormente.
    \item \textbf{RF-1.3 Conversion del modelo.} Se debe de poder convertir el modelo entrenado, el cuál posee un formato \textit{weights} a uno formato apto de la librería TensorFlow.
    \end{itemize}
    \item \textbf{RF-2 Detección de objetos}
    \begin{itemize}
      \tightlist
      \item \textbf{RF-2.1 Detección de objetos sobre una imagen.} Permitir el reconocimiento de los objetos que posee el modelo, a través de una imagen, guardar el resultado de la detección en un vídeo y un fichero CSV con las posiciones.
      \item \textbf{RF-2.2 Detección de objetos sobre un vídeo.} Permitir el reconocimiento de los objetos que posee el modelo, a través de una vídeo, guardar el resultado de la detección en un vídeo y un fichero CSV con las posiciones.
      \item \textbf{RF-2.3 Detección de objetos visualizados por la webcam.} Permitir el reconocimiento de los objetos que posee el modelo, a través de el vídeo que captura la webcam, guardar el resultado de la detección en un vídeo y un fichero CSV con las posiciones
      \item \textbf{RF-2.4 Contabilización de objetos sobre un vídeo.} Permitir la contabilización de los objetos que posee el modelo, a través de una vídeo, , guardar el resultado de la contabilización en un vídeo y un fichero CSV con las posiciones.
    \end{itemize}
\end{itemize}

\subsection{Requisitos no funcionales}
\begin{itemize}
\item \textbf{RNF-1 Usabilidad.} La plataforma debe de ser fácil tanto de aprender a utilizar como clara a la hora de reportar los errores que se puedan cometer. La interfaz debe ser intuitiva.
\item \textbf{RNF-2 Rendimiento.} La interfaz web debe de tener unos tiempos de carga razonables.
\item \textbf{RNF-3 Escalabilidad.} La plataforma debe soportar que se le añadan nuevas funcionalidades con relativa facilidad.
\item \textbf{RNF-4 Disponibilidad.} La plataforma debe de ser accesible a través de Internet sin importar la geolocalización del cliente.
\item \textbf{RNF-5 Mantenibilidad.} La plataforma debe cumplir los estándares de código de cada uno de los lenguajes en los que se desarrolla. 
\item \textbf{RNF-6 Soporte.} La plataforma debe dar soporte a ficheros CSV, MP4, JPG como mínimo. Así como ser compatible con HTML5.
\end{itemize}

\section{Especificación de requisitos}


