\capitulo{7}{Conclusiones y Líneas de trabajo futuras}

En este apartado se van a agrupar tanto las conclusiones a las que se llega tras el desarrollo del proyecto, como los posibles puntos de mejora, como las líneas de trabajo futuras.

\section{Conclusiones}
Conclusiones a las que se llega tras el posterior desarrollo del proyecto.
\begin{itemize}
    \item Los objetivos del proyecto se han cumplido satisfactoriamente
    \begin{list}{\textbullet}{ %
        \addtolength{\itemsep}{-2mm} %
        \setlength{\itemindent}{2mm}}
        \item \texttt{Object Detection}. Posee un algorimto para poder entrenar cualquier modelo con el algoritmo YOLO en la versión 4. Así como, un script para convertir dicho modelo a uno apto para Tensorflow, y así poder detectar vídeos, imágenes y/o trabajar con Obejct Tracking.
        A su vez, podemos exportar las posiciones de cada uno de estos scripts a un fichero \textit{csv}, como guardar el resultado deseado en un vídeo o en una imagen.
        Pero sobre todo, nos permite medir la caldiad del modelo entrenado, con el objetivo de evaluar su calidad de evalaución.  
    \end{list}
    \item La lectura de artículos científicos ha sido algo prácticamente nuevo, siendo necesarias numerosas horas para su interpretación, se debe destacar que según se avanzaba en el desarrollo del proyecto, la familiarización con éstos ha sido satisfactoria, permitiendo asimilar la información como la idea que se iba a implementar.
    \item El trabajar con Object Detection, a través del algoritmo de detección de YOLO, permite que sea rápido y su calidad de detección sea bastante buena.
    \item La creación de un modelo de detección, así como su respectivo dataset de entremiento ha permitido aumentar el conocmiento que se tenia sobre la detección de objetos, así cmo trabajar con herramientas que al inicio del proyecto eran totalmente desconcoidas. 
\end{itemize}

Como \textbf{conclusión} cabe destacar, la oportunidad tanto de aprendizaje como la de poder trabajar en un proyecto nuevo ha supuesto. Desde el inicio ha sido necesario dominar ciertas tecnologías vistas (y nuevas) a lo largo de todo el Grado, incluso a la hora de escribir la documentación.

\section{Líneas de trabajo futuras}