\capitulo{5}{Aspectos relevantes del desarrollo del proyecto}

En esta sección van a ser detallados los aspectos más relevantes que han tenido lugar a lo largo del desarrollo del proyecto. Al tratarse de un desarrolo \textit{software}, y a lo largo de este se han encontrado diferentes retos e inconvenientes. A su vez, se han aplicado buenas prácticas con el objetivo de obtener un resultado de calidad.

\section{Investigación}
El eje principal que posee el proyecto es el \textit{Object Detection}, el cuál a pesar de ser conocido, nunca se había trabajado con él, se puede decir que su conociemnto era puramente teórico. En la misma línea, se desconcoia completamente el funcionamiento de la \textit{Jetson Nano}, la cuál a trabajar de forma intenra con \textit{Ubuntu} ha supuesto
una gran ayuda y beneficio al ser un sistema operativo previamente conocido.
Por todo esto, ha supuesto un doble esfuerzo, especialmente por el primero de ellos, ya que se ha requerido una formación previa par poder conocer el correcto funcionamiento a la hora de entrenar un modelo de detección, así como averiguar cuál era la mejor opción de cara al proyecto.

\clearpage
\section{Metodología \textit{Scrum}}
Tal y coo se comento en el punto \ref{scrum}, el proyecto se ha reliazado siguiendo una metodología ágil, lo cuál nos permite trabajar con \textit{sprints}, de tal manera, que el trabajo a relaizar en cada uno de ellos, se encuentre documentado desde el inicio, para así poder con una mayor eficiencia, pudiendo priorizar las tareas en función de las existentes y contando con diferentes versiones según a la vez que se va siguiendo el desarrollo del proyecto.

Los conocimeientos que se poseían de \textit{Scrum} eran más bien teóricos, pero si que se había tenido la oportunidad de trabajar con ella, durante la realización de las prácticas en empresa.

Una de las principales dificultades encontradas, ha sido la estimación del tiempo que va a ser necesario para pdoer llevarlo a cabo, todo ello apartir de un nombre y descripción, lo cuál ha concluido con resultados no muy exactos, ya que se en algunos casos la estiamción se ha realizado por debajo y en otros por arriba. Si a esto se le suma que ciertas tareas han llevado más tiempo del debido a problemas versiones y de compatibilidades entre sistemas, hay que suamrle que al contar con un determiando número de asiganturas
más la realización de manera paralela de las prácticas, ha conllevado que tareas que de normal llevan pcoo tiempo, han llevado más de lo planeado.

\section{Creación del dataset de entrenamiento y entremiento del modelo de detección}
A lo largo del Grado, no se ha visto nada relacionado con la \textbf{Inteligencia Artificial}, siendo esto una desventaja a la hora de crear el dataset del modelo y etiquetarlo con las características corrspondientes al algoritmo de detección escogido.
