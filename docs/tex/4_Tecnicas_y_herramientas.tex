\capitulo{4}{Técnicas y herramientas}

\section{Metodología}
A lo largo del proyecto se intentado seguir la \textit{metodología ágil Scrum}, pero adaptada ya que para pdoer aplciar esta metodología es necesario contar con un equipo, en el cuál los diferentes miembros se reparten las roles entre los diferentes miembros que lo conforman. En este caso, lso roles recaeen todo sobre una única persona.

\imagen{scrum}{Pasos de la metodología Scrum}

En primer lugar se encuentra el \textit{Product Backlog} \cite{scrum} que se trata del alcance del proyecto, el cuál va variando dependiendo de los \textit{feedbacks} que se van obteniendo en cada \textit{sprint}.

Seguidamente, se encuentra el \textit{Sprint Backlog}, dónde se marcan los requerimientos que deben de alcanzar durante el \textit{sprint} que se va a iniciar, es decir, se trata de acortar las tareas de cada uno 
de los \textit{sprints}.

La siguiente etapa es el \textit{Sprint}, en la cuál tiene lugar la planificación, la implementación, revisión y retrospectiva de la nueva característica software.
Esta etapa suele tener una duración de una a dos semanas.

Como último paso del proceso, se enecuentra el \textit{incremento del producto}, esta fase consiste en tener una reunión con el cliente con la nueva característica en funcionamiento con el objetivo de obtener una \textit{retroalimentación} por parte del cliente y así volver a empezar el proceso de nuevo.


\section{Lenguaje de programación}
A la hora de empezar un nuevo proyecto es importante relacionado con el \textit{Machine Learning} y el \textit{Edge Computing}, es muy importante seleccionar el lenguaje, con el cuál queremos trabajar destacando dos: \textbf{Python} \cite{python} y \textbf{Matlab} \cite{matlab}.
En este caso se decantó por el uso de \textit{Python}, debido al mayor conocimiento de este lenaguaje y haber trabajado más con este lenguaje que con \textit{Matlab}.
No hay grandes ventajas entre escoger uno u otro.


\section{Algoritmo de detección}
A parte de tener claro el lenguaje que se quiere utilizar, otra característica a tener en cuenta es elegir el \textit{algoritmo de detección} en el que se va a basar el modelo.
Existen diferentes algoritmos:
\begin{list}{\textbullet}{ %
    \addtolength{\itemsep}{-2mm} %
    \setlength{\itemindent}{2mm}}

    \item CNN \cite{cnn}
    \item R-CNN \cite{r-cnn}
    \item Faster R-CNN \cite{faster_rcnn}
    \item YOLO \cite{yolo}

\end{list}