\capitulo{1}{Introducción}

Hoy en día la tecnologia ha avanzado tanto, que es muy fácil contar con ella a la hora de realizar ciertas tareas, pero cada vez la demandamos 
más para poder trabajar codo con codo con ella, es decir, recibir su ayuda de tal forma que les podamos asignar tareas asegurando que tendrán un
porcentaje de acierto igual o superior al que tendría si lo realizasemos cualquiera de nosotros.

Pero generalmente, para poder llevar a cabo estas tareas, se necesitan dispositivos con una gran cantidad de computo, ya que necesitaremos entrenarlo 
con el objeto u objetos a predecir, siendo está la tarea más importante y la que más capacidad de computo va a necesitar y la que más recursos va consumir.
Tras su entrenamiento, volveremos a consumir recursos para su detección, de tal forma que necesitaremos un equipo lo suficientemente potente para poder realizar
ambas tareas con efectividad y poder obtener buenos resultados.

Debido a esto, el poder entrenar el modelo en un ordenador lo suficientemente potente y seguidamente poder adaptarlo para poder ser utilizado en dispositivos pequeños
como puede ser la Jetson Nano de NVIDIA, y que este dispositivo lo ejecute, sacrificando el porcentaje de acierto pero respetando los tiempos de ejecución, puede facilitar a
muchos trabajadores y/o investigadores en sus trabajos ya que pueden tener una herramienta fucnional en poco espacio y además fácil de transportar para poder usarla en diferentes
lugares.


