\capitulo{1}{Introducción}

Hoy en día la tecnologia ha avanzado tanto, que es muy fácil contar con ella a la hora de realizar ciertas tareas, pero cada vez la demandamos 
más para poder trabajar codo con codo con ella, es decir, recibir su ayuda de tal forma que les podamos asignar tareas asegurando que tendrán un
porcentaje de acierto igual o superior al que tendría si lo realizasemos cualquiera de nosotros.

Pero generalmente, para poder llevar a cabo estas tareas, se necesitan dispositivos con una gran cantidad de computo, ya que necesitaremos entrenarlo 
con el objeto u objetos a predecir, siendo está la tarea más importante y la que más capacidad de computo va a necesitar y la que más recursos va consumir.
Tras su entrenamiento, volveremos a consumir recursos para su detección, de tal forma que necesitaremos un equipo lo suficientemente potente para poder realizar
ambas tareas con efectividad y poder obtener buenos resultados.

Debido a esto, el poder entrenar el modelo en un ordenador lo suficientemente potente y seguidamente poder adaptarlo para poder ser utilizado en dispositivos pequeños
como puede ser la Jetson Nano de NVIDIA, y que este dispositivo lo ejecute, sacrificando el porcentaje de acierto pero respetando los tiempos de ejecución, puede facilitar a
muchos trabajadores y/o investigadores en sus trabajos ya que pueden tener una herramienta funcional en poco espacio y además fácil de transportar para poder usarla en diferentes
lugares.

\clearpage

\section{Estructura de la memoria}
La memoria consta de las siguiente estructura:
\begin{list}{\textbullet}{ %
    \addtolength{\itemsep}{-2mm} %
    \setlength{\itemindent}{2mm}}

    \item \textbf{Introducción:} establece el contexto inicial entorno a la idea que se va a desarrollar, además de la estructura del documento y de los materiales que se van a entregar.
    \item \textbf{Objetivos del proyecto:} objtivos que se desean alcanzar durante el desarrollo del proyecto.
    \item \textbf{Conceptos teóricos:} exponer los conceptos que son necesarios disponer para llevar a cabo el proyecto.
    \item \textbf{Técnicas y herramientas:} muestras las técnicas y las herramientas que se han utilizado durante el desarrollo del proyecto.
    \item \textbf{Aspectos relevantes del desarrollo del proyecto:} recopilación de los aspectos más representativos que han tenido lugar durante el desarrollo del proyecto.
    \item \textbf{Trabajos relacionados:} presentación de trabajos que se encuentran relacionados de manera destacable con el desarrollo o el concepto del proyecto.
    \item \textbf{Conclusiones y líneas de trabajo futuras:} descripción de las conclusiones obtenidas durante la realización del proyecto y tras la misma, así como las posibles líneas de mejora.
\end{list}

Además, junto a la presente memoria se incluyen los siguientes anexos relacionados con el desarrollo del modelo de detección y su correspondiente prueba en el dispositivo de Edge Computing:
\begin{list}{\textbullet}{ %
    \addtolength{\itemsep}{-2mm} %
    \setlength{\itemindent}{2mm}}

    \item \textbf{Plan de Proyecto Software:} presentar la planificación temporal llevada a cabo durante el desarrollo del proyecto, así como un estudio de la viabilidad del desarrollo.
    \item \textbf{Especificación de Requisitos:} describir de forma detallada lso objetivos generales y los objetivos del proyecto llevado a cabo.
    \item \textbf{Especificación de diseño:} presentar el diseño final del modelo, describiendo el diseño de datos, procedimental y arquitectónico del desarrollo.
    \item \textbf{Documentación técnica de programación:} en este apartado se describen los conocimientos técnicos más relevantes del proyecto, los cuáles son necesarios para poder continuar con el desarrollo.
    \item \textbf{Documentación de usuario:} apartado dirigido al usuario final, dónde se describen los requisitos necesarios en un dispositivo para poder utilizar la herramienta, la instalación de cada uno de ellos, y un manual de usuario, en el que se mostrarán todas las posibles opciones que dispone la herramienta.
\end{list}