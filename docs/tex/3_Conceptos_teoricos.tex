\capitulo{3}{Conceptos teóricos}

Para la compresion de este proyecto, se deben conocer los siguientes conceptos:

\section{Deep Learning} 

El deep learning \cite{deepLearning} es una rama del MachineLearning, donde los algoritmos inspirados en el funcionamiento del cerebro humano (redes neurnales) aprenden a partir de 
grandes cantidades de datos y tratan con un alto número de unidades computacionales.

\section{Edge Computing}

El edge computing \cite{edgeComputing} es un tipo de informática que ocurre, en la ubicación fisica del usuario, en la ubicación de la fuente de los datos o cerca de estas. Permitiendo 
que los usuarios obtengan servicios mas rápidos y fiables.

\section{Raspberry Pi}

Una Raspberry Pi \cite{raspberry} es un mini PC de bajo coste, el cual cabe en una mano. Se encuentra compuesto por un SoC, CPU, RAM,puertos de entrada y salida de audio y vídeo, conectividad de red, 
ranura SD para almacenamiento, reloj, una toma para la alimentación, conexiones para periféricos de bajo nivel..., pero a diferencia de un ordenador común no tiene un botón para endecerla/apagarla.

\section{Object Detection}

El Object Detection \cite{objectDetect} es un técnica de vision por ordenador que permite localizar imágenes y/o vídeos. Estos algoritmos se aprovechan del aprendizaje automático o del profundo 
con el objetivo de obtener resultados significativos, es decir, intentan replicar la inteligencia humana a la hora de reconcoer un objeto.