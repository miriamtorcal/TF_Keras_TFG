\capitulo{3}{Conceptos teóricos}

Para la compresion de este proyecto, se deben conocer los siguientes conceptos:

\section{Deep Learning} 

El Deep Learning \cite{deepLearning} es una rama del Machine Learning, donde los algoritmos inspirados en el funcionamiento del cerebro humano (redes neurnales) aprenden a partir de 
grandes cantidades de datos y tratan con un alto número de unidades computacionales.

Gracias a la neurociencia, el estudiio de casos clínicos de daño cerebral sobrevenidoy los avances en diagnóstico por imágenes sabemos que hay centros específicos del lenguaje, que 
existen redes especializadas en detectar diferentes aspectos de la visión, como los bordes, la simetría, áreas relacionadas con el reconocimiento de rostros y las expresiones emociales de los mismos.
Los módelos de Deep Learning imitan estas características de arquitectura del sistema nervioso, permitiendo que dentro del sistema global haya redes de unidades de proceso que se especialicen en la detección
de determinadas características que se encuentran ocultas en los datos. Dicho enfoque, ha permitido obtener mejores resultados si los comparamos con la redes monolíticas de neuronas artificiales

\imagen{deep_learning_network}{Red neuronal convolucional}

\section{Edge Computing}

El Edge Computing \cite{edgeComputing} es un tipo de informática que ocurre, en la ubicación fisica del usuario, en la ubicación de la fuente de los datos o cerca de estas. Permitiendo 
que los usuarios obtengan servicios mas rápidos y fiables.

La ventaja fundamental de esto, es que permite a las empresas analizar los datos que sean importantes casi en tiempo real, un hecho que en áreas como la fabricación, la sanidad,
las telecomunicaciones o la industria financiera, es una necesidad latente y continua.

Las necesidades industriales hacen que esta tecnología cada vez sea más demandada, debido a que en ciertos entornos la única forma de poder automatizar más los procesos, consiste en tratar
de evitar lo máximo posible la comunicación con la nube, consiguiendo reducir las latencias, consumir menos ancho de banda y por su puesto acceder de manera inmediata a análisis y evaluación
del estado los sensores y dispositivos que la constituyen.

\imagen{edge_computing}{Estructura Edge Computing}

\section{Jetson Nano}

Una Jetson Nano \cite{jetsonNano} es un mini PC de bajo coste, el cual cabe en una mano. Se encuentra compuesto por un SoC, procesador ARM de 64 bits de 4 núcleos y una GPU con arquitectura Maxvell con 128 núcleos de procesameinto gráfico, conectividad de red, 
contando con una potencia total de 472 Gflops, cuenta a su vez, con puertos USB-A, salidas de vídeo HDMI y DisplayPort y un puerto para su conexión a Internet.

\imagen{jetson_nano}{Jetson Nano}

\section{YOLO}

You Only Look Once (YOLO) \cite{yolov4} es un algoritmo de detección que usa Deep Learning y CNN para ello, como su nombre indica sólo necesita mirar la imagen una única vez, de tal forma que la detección es mucho más rápida que en otros algoritmos, pero a cambio de sacrificar rendiemiento a la hora de predecir.
Para llevar a cabo la detección, divide la imagen en una cuadríucla de SxS (imagen de la izquierda). Por cada cuadríucla, predice N posibles "bounding boxes" y calcula la probabilidad de cada una de ellas, es decir, en total se predicen SxSxN cajas diferentes (la gran mayoria con una probabilidad muy baja) (imagen del centro). 
Por último, se eliminan las cajas que estan por debajo de un límite, conocido este como non-max-suppression, de tal forma, que se eliminan los objetos detectados por duplciado, dejando los que poseen un mayor valor de predicción (imagen de la derecha).

\imagen{yolo}{Explicación YOLO}

Para entrenar un modelo, basado en el algoritmo YOLO, tendremos que tener las imégenes con las que vamos a entrenar etiquedas con un contenido como el siguiente:

\imagen{formato_yolo}{Representación del contenido etiquetado YOLO}

El primer parámetro representa el id de la clase, es decir, cuando se etiquetan las imágenes se creara un fichero llamado classes.txt, con los nombres de todas las clases que se han etiquetado para el modelo.
El segundo representa la distancia desde la coordenada 'x' al centro, mientras que el tercero hace lo propio desde la coordenada 'y'.
El cuarto parámetro representa el ancho de la anotación, es decir, el ancho del recuadro que conforma la anotación, el último párametro representa el alto de la anotación.

\clearpage

\section{Object Detection}

El Object Detection \cite{objectDetect} es un técnica de vision por ordenador que permite localizar imágenes y/o vídeos. Estos algoritmos se aprovechan del aprendizaje automático o del profundo 
con el objetivo de obtener resultados significativos, es decir, intentan replicar la inteligencia humana a la hora de reconcoer un objeto.

\imagen{object_detection}{Detección de objetos}