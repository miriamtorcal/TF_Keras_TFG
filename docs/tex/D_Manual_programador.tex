\apendice{Documentación técnica de programación}

\section{Introducción}
En este apéndice van a describirse de forma detallada la documentación técnica de programación. Se describirá la estructura de directorios que posee, la instalación y ejecución, así como las pruebas que se han llevado a cabo. 

El repositorio del proyecto puede ser consultado en el siguiente enlace: \textbf~\url{https://github.com/mtc1003/TF_Keras_TFG}

\section{Estructura de directorios}
\begin{itemize}
    \tightlist
    \item \texttt{/}: es la raíz del proyecto dónde se encuentran tanto el README, la licencia y las carpetas contenedoras del código, documentación y las pruebas previas.
    \item \texttt{/codigo}: es la carpeta que contiene todo el código funcional del proyecto.
    \item \texttt{/codigo/checkpoints}: es la carpeta contenedora de los modelos de detección en formato Tensorflow, Tensorflow Lite y Tensor-RT.
    \item \texttt{/codigo/checkpoints/custom-416}: carpeta que contiene el modelo de detección de las matrículas, que posee el tamaño 416.
    \item \texttt{/codigo/checkpoints/custom-416/saved\_model.pb}: modelo en formato Tensorflow(.pb) de las matrículas.
    \item \texttt{/codigo/checkpoints/heads-416}: carpeta con el modelo de las cabezas en formato Tensorflow.
    \item \texttt{/codigo/checkpoints/heads-416/keras\_metadata.pb}: punto de control del modelo de conversión a .pb.
    \item \texttt{/codigo/checkpoints/heads-416/saved\_model.pb}: modelo detector de cabezas en formato Tensorflow(.pb)
    \item \texttt{/codigo/checkpoints/yolov4-416}: carpeta con el modelo oficial de YOLOv4 en formato Tensorflow.
    \item \texttt{/codigo/checkpoints/yolov4-416/keras\_metadata.pb}: punto de control del modelo de conversión a .pb.
    \item \texttt{/codigo/checkpoints/yolov4-416/saved\_model.pb}: modelo detector de YOLOv4 en formato Tensorflow(.pb)
    \item \texttt{/codigo/checkpoints/custom\_tfl-416}: carpeta con el modelo de detección de las matrículas en formato Tensorflow previo a la conversióna TensorFlow Lite.
    \item \texttt{/codigo/checkpoints/custom\_tfl-416/saved\_model.pb}: modelo detector de matrículas en formato Tensorflow(.pb) preparado para su conversión a .tflite.
    \item \texttt{/codigo/checkpoints/custom\_tflv2-416}: carpeta con el modelo de detección de las matrículas en formato Tensorflow previo a la conversióna TensorFlow Lite.
    \item \texttt{/codigo/checkpoints/custom\_tflv2-416/saved\_model.pb}: modelo detector de matrículas en formato Tensorflow(.pb) preparado para su conversión a .tflite.
    \item \texttt{/codigo/checkpoints/custom-416-int8.tflite}: modelo de detección de las matrículas en formato TensorFlow Lite(.tflite).
    \item \texttt{/codigo/checkpoints/custom-416v2.tflite.tflite}: modelo de detección de las matrículas en formato TensorFlow Lite(.tflite)
    \item \texttt{/codigo/checkpoints/models\_trt.txt}: fichero con los enlaces de los modelos de TensorRT.
    \item \texttt{/codigo/core}: carpeta con los ficheros de configuración utilizados durante el proyecto.
    \item \texttt{/codigo/core/backbone.py}: fichero de Python que contine las funciones relacionadas con la red YOLOv4
    \item \texttt{/codigo/core/commom.py}: fichero de Python que contine la clase BatchNormalization, para los ajustes de la red YOLOv4
    \item \texttt{/codigo/core/config.py}: fichero de Python que contine permite la selección de los ficheros de etiquetas de cara al uso del modelo
    \item \texttt{/codigo/core/functions.py}: fichero de Python que contine las funciones utilizadas en a lo largo de la detección de los objetos.
    \item \texttt{/codigo/core/utils.py}: fichero de Python que contine las funciones relacionadas con la red YOLOv4.
    \item \texttt{/codigo/core/yolov4.py}: fichero de Python que retorna el modelo de YOLO correspondiente.
    \item \texttt{/codigo/data}: carpeta que contine la información necesaria para la detección.
    \item \texttt{/codigo/data/anchors}: carpeta que contiene los anchors de lass diferentes redes.
    \item \texttt{/codigo/data/anchors/basline\_anchors.txt}: fichero de anchors.
    \item \texttt{/codigo/data/anchors/basline\_tiny\_anchors.txt}: fichero de anchors tiny.
    \item \texttt{/codigo/data/anchors/yolov3\_anchors.txt}: fichero de anchors yolov3.
    \item \texttt{/codigo/data/anchors/yolov3\_anchors.txt}: fichero de anchors yolov4.
    \item \texttt{/codigo/data/classes}: carpeta que contiene los fichero de etiquetas de los diferentes modelos.
    \item \texttt{/codigo/data/classes/coco.names}: fichero con las etiquetas del modelo oficial de YOLOV4.
    \item \texttt{/codigo/data/classes/custom.names}: fichero con las etiquetas del modelo de detección de matrículas.
    \item \texttt{/codigo/data/classes/heads.names}: fichero con las etiquetas del modelo de detección de las cabezas.
    \item \texttt{/codigo/data/classes/voc.names}: fichero de etiquetas del modelo de voc.
    \item \texttt{/codigo/data/classes/yymnist.names}: fichero de etiquetas del modelo de yymnist, detector de números.
    \item \texttt{/codigo/data/dataset}: carpeta que contiene los ficheros etiquetados a la hora de evaluar un modelo (ruta de la imagen posición detectada y valor de la clase).
    \item \texttt{/codigo/data/dataset/head.txt}: fichero de evaluacion del módelo de las cabezas. 
    \item \texttt{/codigo/data/dataset/license\_plate.txt}: fichero de evaluación del modelo de las matrículas.
    \item \texttt{/codigo/data/dataset/val2017.txt}: fichero de evaluación del modelo coco.
    \item \texttt{/codigo/data/images}: carpeta que contiene diferentes imagenes para su detección.
    \item \texttt{/codigo/data/video}: carpeta que contien diferentes vídeos para su detección/contabilización.
    \item \texttt{/codigo/deep\_sort}: carpeta que contiene los diferentes ficheros en Python para su evaluacion con Object Tracking.
    \item \texttt{/codigo/deep\_sort/detection.py}: fichero Python que contiene las funciones de detección para Obejct Tracking.
    \item \texttt{/codigo/deep\_sort/iou\_matching.py}: fichero Python que tiene las funciones de la maedida iou para Object Tracking.
    \item \texttt{/codigo/deep\_sort/kalman\_filter.py}: fichero Python que contiene el algoritmo del filtro de Kalman\cite{kalman_filter}.  
    \item \texttt{/codigo/deep\_sort/linear\_assignment.py}: Fichero Python que contiene las funciones relacionadas con la asignación linear.
    \item \texttt{/codigo/deep\_sort/nn\_matching.py}: Fichero Python con funciones de ajuste del algorimto de vecinos más cercanos\cite{knn}.
    \item \texttt{/codigo/deep\_sort/preprocessing.py}: Fichero Python con las funciones del preprocesado para Object Tracking.
    \item \texttt{/codigo/deep\_sort/track.py}: fichero Python que contiene las funciones necearias para detectar los objetos y sus etiquetas correspondeitnes, con su respectivo número de identificación.
    \item \texttt{/codigo/deep\_sort/tracker.py}: fichero Python que contiene las funciones necearias para detectar los objetos y sus etiquetas correspondeitnes, con su respectivo número de identificación.
    \item \texttt{/codigo/detections}: carpeta con el lso resultados de las detecciones obtenidas mediante la línea de comandos.
    \item \texttt{/codigo/detections/images}: carpeta con el resultado de las imagenes detectadas.
    \item \texttt{/codigo/detections/videos}: carpeta con el resultado de los vídeos detectados.
    \item \texttt{/codigo/mAP}: carpeta que contiene lso resultados de la evaluaciones de los modelos, así como scripts de ayuda para ello.
    \item \texttt{/codigo/mAP/extra}: carepeta con los cripts de ayuda para la evaluación del modelo.
    \item \texttt{/codigo/mAP/extra/intersect-gt-and-pred.py}: fichero Python que calcula la intersección entre la posición real del objeto y la obtenida por el modelo, con el objetivo de evaluar la calidad del modelo.
    \item \texttt{/codigo/mAP/extra/remove\_space.py}: fichero Python que elimina lso espacios de las etiquetas de las clases de los modelos.
    \item \texttt{/codigo/mAP/ground-truth}: carpeta que contiene los ficheros .txt de cada imagen a evalaur con sus posiciones originales en formato YOLO, junto el nombre de la etiqueta que le corresponde.
    \item \texttt{/codigo/mAP/predicted}: carpeta que contiene los ficheros .txt de cada imagen a evaluar con sus posiciones detectadas en formato YOLO, junto el nombre de la etiqueta que le corresponde.
    \item \texttt{/codigo/mAP/results\_custom\_tf\_complete}: carpeta con los resultados de la evalaución del modelo de las matrículas.
    \item \texttt{/codigo/mAP/results\_heads\_tf\_complete}: carpeta con los resultados de la evalaución del modelo de las cabezas.
    \item \texttt{/codigo/mAP/main.py}: fichero Python que representa el resultado de la evalaución del modelo.
    \item \texttt{/codigo/model\_data}: carpeta que contiene el modelo mars-small128.pb, utilizado en la inicialización de Obejct Tracking.
    \item \texttt{/codigo/static}: carpeta que contiene los ficheros 'estaticos' para Flask.
    \item \texttt{/codigo/static/css}: carpeta que contiene los diferentes ficheros de estilos\cite{css} usados a lo largo de la app Flask.
    \item \texttt{/codigo/static/js}: carpeta que contiene los diferentes scripts de JavaScript\cite{js} utilizados a lo largo de la app Flask.
    \item \texttt{/codigo/static/detections}: carpeta que contiene las imagenes, videos etiquetados tras su detección, así como los ficheros CSV de las posiciones.
    \item \texttt{/codigo/static/imgs}: carpeta con todas las imagenes usadas a lo largo de la app Flask.
    \item \texttt{/codigo/temp}: carpeta que almacena los ficheros de detección temporales, generados al inicio de las detecciones en la app Flask.
    \item \texttt{/codigo/templates}: carpeta que contienelos ficheros .html usados a lo alrgo de la app Flask.
    \item \texttt{/codigo/tools}: carpeta que contiene los scripts Python utilizados cómo herramientas a la hora de detectar.
    \item \texttt{/codigo/tools/freeze\_model.py}: script Python que convierte el gráfico del modelo de TensorFlow a uno con extensión .pb.
    \item \texttt{/codigo/tools/generate\_detections.py}: script Python que obtiene las 'cajas' en las cuáles se encuentran los objetos qu ehan sido detectados por el modelo.
    \item \texttt{/codigo/train}: carpeta que tiene los scripts de Python y de GoogleColab, así como los ficheros neecsarios para llevar a cabo el entrenamiento de un modelo de YOLOv4.
    \item \texttt{/codigo/trt}: carpeta que contiene el script de GoogleColab de conversión del fichero de pesos de YOLOv4 (.weights) a un modelo de TensorRT.
    \item \texttt{/codigo/app.py}: fichero de Python que es la propia app de Flask.
    \item \texttt{/codigo/convert\_tflite.py}: fichero de Python que convierte el modelo deseado a uno de TensorFlow Lite.
    \item \texttt{/codigo/convert\_trt.py}: fichero de Python que convierte el modelo deseado a uno de TensorRT.
    \item \texttt{/codigo/detect.py}: fichero de Python que detecta objetos en una imagen, según un modelo de detección.
    \item \texttt{/codigo/detectVideo.py}: fichero de Python que detecta objetos en una vídeo, según un modelo de detección.
    \item \texttt{/codigo/evaluate.py}: fichero de Python que evalua un modelo de detección, con el objetivo de medir su calidad a l hora de predeccir.
    \item \texttt{/codigo/objectTracker.py}: fichero de Python que contabiliza objetos en un vídeo, según un modelo de detección.
    \item \texttt{/codigo/preprocessDataEvaluate.py}: fichero de Python que obtiene las posiciones de las imágenes en el formato necesario apra su evalaución.
    \item \texttt{/codigo/save\_model\_tflite.py}: fichero de Python que convierte un fichero de pesos en formato .weights a un modelo de TensorFlow Lite. 
    \item \texttt{/codigo/save\_model.py}: fichero de Python que convierte un fichero de pesos en formato .weights a un modelo de TensorFlow.  
\end{itemize}

\section{Manual del programador}
En esta subsección se describen todos los recursos utilizados para poder llevar a cabo el proyecto.
De tal forma que un futuro desarrollador/mantenedor del proyecto no tenga inconvenientes a la hora de retomar el proyecto y conocerlo.

\subsection{Entorno de desarrollo}
Para poder continuar con el desarrollo del proyecto, será necesario contar con el siguiente \textit{software} instalado en el equipo:
\begin{itemize}
    \tightlist
    \item Python 3.7
    \item Bibliotecas de Python
    \item VSCode
    
\end{itemize}

A continuación, se comentára de forma detallada la instalación de los diferentes requerimientos.

\subsection{Python 3.7}
La versión de Python 3.7, se encuentra dispnible desde \cite{pythonDownload}. Es muy importante que los ficheros binarios se encuentren en el PATH del sistema, para evitar así posibles problemas de ejecución.

\subsection{Bibliotecas de Python}
Este punto, es de los más importantes para hacer funcionar el proyecto, ya que son necesarias unas determiandas librerías y en unas versiones concretas, para que todo se integre correctamente y así funcione todo
cómo un sistema homogéneo. Ver Tabla~\ref{tab:bibliotecas-python}

\begin{table}[p]
    \centering
    \begin{tabular}{lcl}
        \toprule
        \textbf{Biblioteca} & \textbf{Versión} & \textbf{Descripción}\\
        \midrule
        \rowcolor[HTML]{EFEFEF} 
        \texttt{absl-py} & 0.15.0 & Creación de aplaicaciones sencillas.\\
        \texttt{flask} & 1.1.2 & Web \textit{framework}.\\ \rowcolor[HTML]{EFEFEF}
        \texttt{numpy} & 1.22.3 & Computación de \textit{arrays}.\\
        \texttt{pandas} & 0.25.1 & Estructuras de datos.\\ \rowcolor[HTML]{EFEFEF}
        \texttt{requests} & 2.27.0 & \textit{Requests} para humanos.\\
        \texttt{keyboard} & 0.13.5 & Interacción del teclado desde Python\\ \rowcolor[HTML]{EFEFEF}
        \texttt{pafy} & 0.5.5 & Recuperar contenido y metadatos de YouTube\\        
        \texttt{youtube-dll} & 2020.12.2 & Descargar vídeos de Youtube junto con su información\\ \rowcolor[HTML]{EFEFEF}
        \texttt{pandas} & 1.3.5 & Potentes estructuras de datos para análisis de datos, series temporales y estadísticas\\
        \texttt{numpy} & 1.21.6 & Paquete de computación matricial\\ \rowcolor[HTML]{EFEFEF}
        \texttt{opencv-python} & 4.1.1.26 & Visión Artificial para el lenguaje Python\\
        \texttt{tensorflow} & 2.8.0 & Framework dw Machine Learning\\ \rowcolor[HTML]{EFEFEF}
        \texttt{tensorflow-gpu} & 2.3.0 & Framework de Machine Learning para GPU\\
        \texttt{pillow} & 9.2.0 & Librería de imágenes\\ \rowcolor[HTML]{EFEFEF}
        \texttt{easydict} & 1.9 & Acceso a los valores de un \textit{dict} como atributos\\
        \texttt{matplotlib} & 3.5.3 & Trazado en Python\\ \rowcolor[HTML]{EFEFEF}
        \bottomrule
    \end{tabular}
    \caption{Bibliotecas utilizadas y sus versiones.}\label{tab:bibliotecas-python}
\end{table}

Las versiones que se indican en la Tabla~\ref{tab:bibliotecas-python}, son las que se han utilizado a lo largo del desarrollo del proyecto, las cuáles pueden ser actualizadas a versiones futuras, siempre y cuando estás sean compatibles entre sí o con las páginas web con las que trabajan por debajo.

\section{Compilación, instalación y ejecución del proyecto}

En esta sección, se va a detallar el proceso a seguir para poder poseer el proyecto en local, y así poder utilizarlo y/o modificarlo. 

\subsection{Adquisición del código fuente}
El primer paso, es la obtención del código en el equipo, para ello podremos seguir una de las siguientes aproximaciones:

\begin{itemize}
    \item Mediante el uso de la terminal.
    \begin{enumerate}
    \tightlist
    \item Apertura de la terminal.
    \item Desplazarse al directorio en donde se desee clonar el repositorio (usando \texttt{cd} en Unix o \texttt{dir} en Windows).
    \item Hacer uso del siguiente comando:\\
    \texttt{git clone https://github.com/mtc1003/TF\_Keras\_TFG.git}
    \item Se dispone de una copia idéntica a la alojada en el repositorio de \texttt{GitHub}.
    \end{enumerate}
    
    \item Descarga desde el navegador.
    \begin{itemize}
    \tightlist
    \item Apertura del navegador de preferencia.
    \item Introducir en la barra de búsqueda la siguiente dirección:\\
    \texttt{https://github.com/mtc1003/TF\_Keras\_TFG/archive/refs/heads/master.zip}
    \item Aceptar la descarga en caso de tener habilitada la comprobación.
    \item Navegar con el Explorador de archivos del sistema hasta el directorio de descarga.
    \end{itemize}

    \item Uso de \texttt{Fork}.
    \begin{itemize}
    \tightlist
    \item Apertura de la aplicación.
    \item Hacer \textit{click} en \textit{File} y del desplegable de opciones seleccionar \textit{Clone}.
    \item Dentro de la ventana de \textit{Clone}:
    \begin{itemize}
    \item En \textit{Respository Url} introducir:\\ https://github.com/mtc1003/TF\_Keras\_TFG.git.
    \item En \textit{Parent Folder} introducir: la ruta en la que se clonará el repositorio en local.
    \item En \textit{Name} introducir: el nombre que recibirá el proyecto en lcoal.
    \end{itemize}
    \item Hacer \textit{click} en \textit{Clone}.
    \end{itemize}
\end{itemize}

\subsection{Creación de entorno virtual de trabajo}
Para poder trabajar con este proyecto (independientemente de si es para desarrollo o producción) hacen falta una serie de bibliotecas concretas de Python (en unas versiones determinadas), las cuáles, como es lógico, deben estar en la máquina en la que se va a ejecutar, es decir, en la que se encuentra el código. 
El proyecto está preparado para crear un entorno de \texttt{Conda} propio, de forma que no interfiera con otros proyectos y sea más sencillo de mantener y actualizar.

Se recomienda que los binarios de anaconda o miniconda estén configurados en el \texttt{path} del sistema para poder utilizar el comando \texttt{conda} desde la línea de comandos.

El proceso de creación del entrono virtual con \texttt{Conda} (trabajar con CPU) es el siguiente:
\begin{enumerate}
\tightlist
\item Apertura de la terminal.
\item Navegar hasta la raíz del proyecto.
\item Crear el entorno con:\\
\texttt{conda env create -f OD\_MTC\_CPU.yml}
\item Cuando se desee utilizar se debe activar:\\
\texttt{conda activate ODMTC}
\end{enumerate}

Por otro lado, el proceso de creación del entrono virtual con \texttt{Conda} (trabajar con GPU) es el siguiente:
\begin{enumerate}
\tightlist
\item Apertura de la terminal.
\item Navegar hasta la raíz del proyecto.
\item Crear el entorno con:\\
\texttt{conda env create -f OD\_MTC\_GPU.yml}
\item Cuando se desee utilizar se debe activar:\\
\texttt{conda activate ODMTC\_GPU}
\end{enumerate}

También se puede utilizar el procedimiento habitual para importar las bibliotecas al actual \texttt{venv} de la sesión de la terminal, pero se desaconseja su uso ya que un entorno 'genérico' antes o después se actualizará por otros proyectos, pudiendo generar incompatibilidades con el proyecto.

\section{Pruebas del sistema}
En está sección se van a describir las pruebas que se realizan con un modelo entrenado previamente, para así concoer su fiabilidad.