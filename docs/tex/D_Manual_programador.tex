\apendice{Documentación técnica de programación}

\section{Introducción}
En este apéndice van a describirse de forma detallada la documentación técnica de programación. Se describirá la estructura de directorios que posee, la instalación y ejecución, así como las pruebas que se han llevado a cabo. 
\section{Estructura de directorios}
\begin{itemize}
    \tightlist
    \item \texttt{/}: es la raíz del proyecto dónde se encuentran tanto el README, la licencia y las carpetas contenedoras del código, documentación y las pruebas previas.
    \item \texttt{/codigo}: es la carpeta que contiene todo el código funcional del proyecto.
    \item \texttt{/codigo/checkpoints}: es la carpeta contenedora de los modelos de detección en formato Tensorflow, Tensorflow Lite y Tensor-RT.
    \item \texttt{/codigo/checkpoints/custom-416}: carpeta que contiene el modelo de detección de las matrículas, que posee el tamaño 416.
    \item \texttt{/codigo/checkpoints/custom-416/saved\_model.pb}: modelo en formato Tensorflow(.pb) de las matrículas.
    \item \texttt{/codigo/checkpoints/heads-416}: carpeta con el modelo de las cabezas en formato Tensorflow.
    \item \texttt{/codigo/checkpoints/heads-416/keras\_metadata.pb}: punto de control del modelo de conversión a .pb.
    \item \texttt{/codigo/checkpoints/heads-416/saved\_model.pb}: modelo detector de cabezas en formato Tensorflow(.pb)
    \item \texttt{/codigo/checkpoints/yolov4-416}: carpeta con el modelo oficial de YOLOv4 en formato Tensorflow.
    \item \texttt{/codigo/checkpoints/yolov4-416/keras\_metadata.pb}: punto de control del modelo de conversión a .pb.
    \item \texttt{/codigo/checkpoints/yolov4-416/saved\_model.pb}: modelo detector de YOLOv4 en formato Tensorflow(.pb)
    \item \texttt{/codigo/checkpoints/custom\_tfl-416}: carpeta con el modelo de detección de las matrículas en formato Tensorflow previo a la conversióna TensorFlow Lite.
    \item \texttt{/codigo/checkpoints/custom\_tfl-416/saved\_model.pb}: modelo detector de matrículas en formato Tensorflow(.pb) preparado para su conversión a .tflite.
    \item \texttt{/codigo/checkpoints/custom\_tflv2-416}: carpeta con el modelo de detección de las matrículas en formato Tensorflow previo a la conversióna TensorFlow Lite.
    \item \texttt{/codigo/checkpoints/custom\_tflv2-416/saved\_model.pb}: modelo detector de matrículas en formato Tensorflow(.pb) preparado para su conversión a .tflite.
    \item \texttt{/codigo/checkpoints/custom-416-int8.tflite}: modelo de detección de las matrículas en formato TensorFlow Lite(.tflite).
    \item \texttt{/codigo/checkpoints/custom-416v2.tflite.tflite}: modelo de detección de las matrículas en formato TensorFlow Lite(.tflite)
    \item \texttt{/codigo/checkpoints/models\_trt.txt}: fichero con los enlaces de los modelos de TensorRT.
    \item \texttt{/codigo/core}: carpeta con los ficheros de configuración utilizados durante el proyecto.
    \item \texttt{/codigo/core/backbone.py}: fichero de Python que contine las funciones relacionadas con la red YOLOv4
    \item \texttt{/codigo/core/commom.py}: fichero de Python que contine la clase BatchNormalization, para los ajustes de la red YOLOv4
    \item \texttt{/codigo/core/config.py}: fichero de Python que contine permite la selección de los ficheros de etiquetas de cara al uso del modelo
    \item \texttt{/codigo/core/functions.py}: fichero de Python que contine las funciones utilizadas en a lo largo de la detección de los objetos.
    \item \texttt{/codigo/core/utils.py}: fichero de Python que contine las funciones relacionadas con la red YOLOv4.
    \item \texttt{/codigo/core/yolov4.py}: fichero de Python que retorna el modelo de YOLO correspondiente.
    \item \texttt{/codigo/data}: carpeta que contine la información necesaria para la detección.
    \item \texttt{/codigo/data/anchors}: carpeta que contiene los anchors de lass diferentes redes.
    \item \texttt{/codigo/data/anchors/basline\_anchors.txt}: fichero de anchors.
    \item \texttt{/codigo/data/anchors/basline\_tiny\_anchors.txt}: fichero de anchors tiny.
    \item \texttt{/codigo/data/anchors/yolov3\_anchors.txt}: fichero de anchors yolov3.
    \item \texttt{/codigo/data/anchors/yolov3\_anchors.txt}: fichero de anchors yolov4.
    \item \texttt{/codigo/data/classes}: carpeta que contiene los fichero de etiquetas de los diferentes modelos.
    \item \texttt{/codigo/data/classes/coco.names}: fichero con las etiquetas del modelo oficial de YOLOV4.
    \item \texttt{/codigo/data/classes/custom.names}: fichero con las etiquetas del modelo de detección de matrículas.
    \item \texttt{/codigo/data/classes/heads.names}: fichero con las etiquetas del modelo de detección de las cabezas.
    \item \texttt{/codigo/data/classes/voc.names}: fichero de etiquetas del modelo de voc.
    \item \texttt{/codigo/data/classes/yymnist.names}: fichero de etiquetas del modelo de yymnist, detector de números.
    \item \texttt{/codigo/data/dataset}: carpeta que contiene los ficheros etiquetados a la hora de evaluar un modelo (ruta de la imagen posición detectada y valor de la clase).
    \item \texttt{/codigo/data/dataset/head.txt}: fichero de evaluacion del módelo de las cabezas. 
    \item \texttt{/codigo/data/dataset/license\_plate.txt}: fichero de evaluación del modelo de las matrículas.
    \item \texttt{/codigo/data/dataset/val2017.txt}: fichero de evaluación del modelo coco.
    \item \texttt{/codigo/data/images}: carpeta que contiene diferentes imagenes para su detección.
    \item \texttt{/codigo/data/video}: carpeta que contien diferentes vídeos para su detección/contabilización.
    \item \texttt{/codigo/deep\_sort}: carpeta que contiene los diferentes ficheros en Python para su evaluacion con Object Tracking.
    \item \texttt{/codigo/deep\_sort/detection.py}: fichero Python que contiene las funciones de detección para Obejct Tracking.
    \item \texttt{/codigo/deep\_sort/iou\_matching.py}: fichero Python que tiene las funciones de la maedida iou para Object Tracking.
    \item \texttt{/codigo/deep\_sort/kalman\_filter.py}: fichero Python que contiene el algoritmo del filtro de Kalman\cite{kalman_filter}.  
    \item \texttt{/codigo/deep\_sort/linear\_assignment.py}: Fichero Python que contiene las funciones relacionadas con la asignación linear.
    \item \texttt{/codigo/deep\_sort/nn\_matching.py}: Fichero Python con funciones de ajuste del algorimto de vecinos más cercanos\cite{knn}.
    \item \texttt{/codigo/deep\_sort/preprocessing.py}: Fichero Python con las funciones del preprocesado para Object Tracking.
    \item \texttt{/codigo/deep\_sort/track.py}: fichero Python que contiene las funciones necearias para detectar los objetos y sus etiquetas correspondeitnes, con su respectivo número de identificación.
    \item \texttt{/codigo/deep\_sort/tracker.py}: fichero Python que contiene las funciones necearias para detectar los objetos y sus etiquetas correspondeitnes, con su respectivo número de identificación.
    \item \texttt{/codigo/detections}: carpeta con el lso resultados de las detecciones obtenidas mediante la línea de comandos.
    \item \texttt{/codigo/detections/images}: carpeta con el resultado de las imagenes detectadas.
    \item \texttt{/codigo/detections/videos}: carpeta con el resultado de los vídeos detectados.
    \item \texttt{/codigo/mAP}: carpeta que contiene lso resultados de la evaluaciones de los modelos, así como scripts de ayuda para ello.
    \item \texttt{/codigo/mAP/extra}: carepeta con los cripts de ayuda para la evaluación del modelo.
    \item \texttt{/codigo/mAP/extra/intersect-gt-and-pred.py}: fichero Python que calcula la intersección entre la posición real del objeto y la obtenida por el modelo, con el objetivo de evaluar la calidad del modelo.
    \item \texttt{/codigo/mAP/extra/remove\_space.py}: fichero Python que elimina lso espacios de las etiquetas de las clases de los modelos.
    \item \texttt{/codigo/mAP/ground-truth}: carpeta que contiene los ficheros .txt de cada imagen a evalaur con sus posiciones originales en formato YOLO, junto el nombre de la etiqueta que le corresponde.
    \item \texttt{/codigo/mAP/predicted}: carpeta que contiene los ficheros .txt de cada imagen a evaluar con sus posiciones detectadas en formato YOLO, junto el nombre de la etiqueta que le corresponde.
    \item \texttt{/codigo/mAP/results\_custom\_tf\_complete}: carpeta con los resultados de la evalaución del modelo de las matrículas.
    \item \texttt{/codigo/mAP/results\_heads\_tf\_complete}: carpeta con los resultados de la evalaución del modelo de las cabezas.
    \item \texttt{/codigo/mAP/main.py}: fichero Python que representa el resultado de la evalaución del modelo.
    \item \texttt{/codigo/model\_data}: carpeta que contiene el modelo mars-small128.pb, utilizado en la inicialización de Obejct Tracking.
    \item \texttt{/codigo/static}: carpeta que contiene los ficheros 'estaticos' para Flask.
    \item \texttt{/codigo/static/css}: carpeta que contiene los diferentes ficheros de estilos\cite{css} usados a lo largo de la app Flask.
    \item \texttt{/codigo/static/js}: carpeta que contiene los diferentes scripts de JavaScript\cite{js} utilizados a lo largo de la app Flask.
    \item \texttt{/codigo/static/detections}: carpeta que contiene las imagenes, videos etiquetados tras su detección, así como los ficheros CSV de las posiciones.
    \item \texttt{/codigo/static/imgs}: carpeta con todas las imagenes usadas a lo largo de la app Flask.
    \item \texttt{/codigo/temp}: carpeta que almacena los ficheros de detección temporales, generados al inicio de las detecciones en la app Flask.
    \item \texttt{/codigo/templates}: carpeta que contienelos ficheros .html usados a lo alrgo de la app Flask.
    \item \texttt{/codigo/tools}: carpeta que contiene los scripts Python utilizados cómo herramientas a la hora de detectar.
    \item \texttt{/codigo/tools/freeze\_model.py}: script Python que convierte el gráfico del modelo de TensorFlow a uno con extensión .pb.
    \item \texttt{/codigo/tools/generate\_detections.py}: script Python que obtiene las 'cajas' en las cuáles se encuentran los objetos qu ehan sido detectados por el modelo.
    \item \texttt{/codigo/train}: carpeta que tiene los scripts de Python y de GoogleColab, así como los ficheros neecsarios para llevar a cabo el entrenamiento de un modelo de YOLOv4.
    \item \texttt{/codigo/trt}: carpeta que contiene el script de GoogleColab de conversión del fichero de pesos de YOLOv4 (.weights) a un modelo de TensorRT.
    \item \texttt{/codigo/app.py}: fichero de Python que es la propia app de Flask.
    \item \texttt{/codigo/convert\_tflite.py}: fichero de Python que convierte el modelo deseado a uno de TensorFlow Lite.
    \item \texttt{/codigo/convert\_trt.py}: fichero de Python que convierte el modelo deseado a uno de TensorRT.
    \item \texttt{/codigo/detect.py}: fichero de Python que detecta objetos en una imagen, según un modelo de detección.
    \item \texttt{/codigo/detectVideo.py}: fichero de Python que detecta objetos en una vídeo, según un modelo de detección.
    \item \texttt{/codigo/evaluate.py}: fichero de Python que evalua un modelo de detección, con el objetivo de medir su calidad a l hora de predeccir.
    \item \texttt{/codigo/objectTracker.py}: fichero de Python que contabiliza objetos en un vídeo, según un modelo de detección.
    \item \texttt{/codigo/preprocessDataEvaluate.py}: fichero de Python que obtiene las posiciones de las imágenes en el formato necesario apra su evalaución.
    \item \texttt{/codigo/save\_model\_tflite.py}: fichero de Python que convierte un fichero de pesos en formato .weights a un modelo de TensorFlow Lite. 
    \item \texttt{/codigo/save\_model.py}: fichero de Python que convierte un fichero de pesos en formato .weights a un modelo de TensorFlow.  
\end{itemize}

\section{Manual del programador}
En esta subsección se describen todos los recursos utilizados para poder llevar a cabo el proyecto.
De tal forma que un futuro desarrollador/mantenedor del proyecto no tenga inconvenientes a la hora de retomar el proyecto y conocerlo.

\subsection{Entorno de desarrollo}
Para poder continuar con el desarrollo del proyecto, será necesario contar con el siguiente \textit{software} instalado en el equipo:
\begin{itemize}
    \tightlist
    \item Python 3.7
    \item Bibliotecas de Python
    \item VSCode
\end{itemize}

A continuación, se comentára de forma detallada la instalación de los diferentes requerimientos.

\section{Compilación, instalación y ejecución del proyecto}

\section{Pruebas del sistema}
